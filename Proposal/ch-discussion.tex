The assumption that coordination requires coercive hierarchy is so pervasive
that it is common to see the words ``hierarchy'' and ``organization'' used
interchangeably, as in the conventional ``org chart.''
Hierarchy is certainly a well-demonstrated tool for achieving coordination,
but that coordination comes with costs.
In terms of principles, hierarchical organizations rely on coercion and power imbalances,
sacrificing egalitarianism.
More practically, when information is distributed but decision-making is centralized,
that decision-making necessarily excludes potentially useful information.
The emergence of successful large-scale, internet-enabled, non-hierarchical collaborations
suggests that perhaps, at leaast in some cases,
neither the principle of egalitarianism nor the practical wisdom of the
crowd needs to be sacrificed to achieve effective coordination and collaboration.
The key question is: how do such large-scale participatory projects achieve success?

The work described and proposed here is motivated by and founded on empirical observations,
but it is also speculative in nature.
By employing agent-based models and experiment, it is possible to understand
not just how things are, but how things could be.
Rather than seeking to identify principles that apply broadly to existing
collaborations, this work starts from empirical observations of collaborations
that have done something unusual and attempts to identify robust principles
that might be applied to transform existing collaborations or create new ones.
The repeated appearance of small interlocking groups in successful large-scale
participatory collaborations suggests that this type of network structure might
be such a principle.
It may seem counterintuitive or even impossible that collaborations and organizations might deliberately
alter their interpersonal network structure.
But in fact, countless practices such as job interviews, letters of recommendation,
performance reviews, internships, mentorships, etc. all serve exactly this purpose;
to deliberately shape the social network within a collaboration.
By analogy, the task of deliberately designing a neighborhood might seem impossible,
but urban planners have made a science of it.

Between the completed chapters of this dissertation, a few commonalities emerge.
While it is conventional to describe the ``quality'' of a collaborive output,
both Chapter \ref{chap:wp-prod-perf} and Chapter{chap:abm} demonstrate that
quality can be multidimensional.
Specifically, Chapter \ref{chap:wp-prod-perf} shows that productivity and performance can
be anti-correlated in some contexts and not in others.
Similarly, Chapter \ref{chap:abm} identifies contexts where altering network structure
and social learning strategies can improve just performance, just productivity, or both.
Another common theme is that subtle differences in task or behavior can have dramatic
influences on output quality.
With regards to network deliberation in particular, Chapter \ref{chap:abm} suggests
that it is most effective in collaborations with a tendancy towards conformity.
Such conformity might occur because of social norms, or to compensate for sparse information.
Interestingly, the correlations between network properties and performance on Wikipedia
were consistent with a conformity-based social learning strategy.
Why network deliberation seems to improve conformity-based collaboration remains an
open question for future work.
And whether network deliberation has a similar effect in a controlled experimental
setting is yet to be seen.

While this work only scratches the surface of factors contributing to successful collaboration,
especially in the messiness of the real world, it identifies previously overlooked distinctions
that may help refine the study of participatory collaboration. Furthermore, starting from empirical
observation of large-scale participatory collaborations, this work has identified, refined, and formalized
network deliberation, and shown that, at least in agent based models, exhibits a distinct influence
on collaborative outcomes, independent of other network properties.
The ongoing work described in Chatper \ref{chap:experiment} will shed light on whether network deliberation
plays a causal role in enabling large-scale participatory collaborations.

