\section{Introduction}

\section{Experimental Design}
Overview

\subsection{Participant Recruitment and Incentives}
Participants will be recruited to discuss a policy issue.
The policy issue will be chosen to be relevant to many individuals,
and to be amenable to a predefined list of proposed solutions,
as well as to be relevant at the time the experiment is conducted.
Example policy questions include
``Where should the use of electric scooters be permitted?'' and
``Given the COVID-19 pandemic, how should schools in the US reopen in Fall 2021?.''
Multiple waves of deliberation may be held using different questions and participants,
depending factors such as participant attrition.

Participants will be recruited using a number of methods.
Participants will be recruited individually through posts on web forums
relevant to the policy question (e.g., reddit),
relevant facebook groups,
and relevant email lists.
Participant pools will also be utilized,
such as lists of previous participants who have expressed interest in future
studies,
and the UMSI Online Recruitment System for Economic Experiments.
Individual participants will also be recruited using promoted posts and
advertisements on Twitter, Facebook, etc.
Separately, participants may also be recruited through partnership with
community groups who wish to try network deliberation as part of an organizational
decision-making strategy.
The number of such collaborations will depend on the number of sufficiently-large
partner organizations that have decisions amenable to network deliberation
over the course of the experiment.

All participants will be paid incentives for participation in the deliberation
and completion of ranked-choice votes and pre/post-experiment surveys.
It is crucial that the incentive structure incentivize participation,
without incentivizing any particular deliberative strategy
(e.g., prefering a specific outcome, maintaining an original position, or
seeking agreement/disagreement).
Any such incentive structure could motivate participants to act based on their
preferences for incentive payments rather than their true preferences regarding
the policy question under consideration,
and disincentivize honest deliberation about the policy question.
As such, incentives will be paid as the same fixed amount to all participants.


\subsection{Experimental Protocol}

At the beginning of the experiment, participants will complete a short
demographic survey.
This survey will enable the identification of sample bias in age, gender, race,
and other demographic factors.

When participants are enrolled, they will be assigned to one of three treatment
groups: control, random-pod, long-path.
The control group will deliberate in a single large group.
The random-pod and long-path treatments will deliberate in a series of small
pods (network deliberation).
These treatments will vary only by the pod assignment method used:
either random-pod or long-path (see Chapter \ref{chap:abm}).

Deliberation will be divided into a predetermined number of {\em rounds}, each lasting
a fixed length of time.
Current plans are to use three stages of two days each,
although the specifics and timing of a policy quesiton may require altering these
numbers.
Before and after each stage, participants will be asked to rank the proposed
solutions to the policy question according to their individual preference.
During each round, participants will be shown a discussion prompt and will be
able to post and reply to other participants.
Example prompts are shown in Tabel \ref{tab:prompts}.
Participants will only be able to see or interact with posts from others in
their current pod.
(all control group participants will be able to interact with all others
at all times).
Participants will be able to view posts and comments that were visible to them
in previous rounds, but will no longer be able to interact with or reply to
those posts and comments.

\begin{table}
\center
\label{tab:prmpts}
\begin{tabular}{|p{0.3in}|p{3.6in}|}
\hline
Stage & Prompt \\
\hline
1 &
Which proposals do you prefer, and why?
If there are disagreements, try asking questions to understand the perspective
of others and to understand the causes of the disagreement.
\\
\hline
2 & In the previous round of discussion, what opinions and reasoning did you
observe in your group?
How much agreement was there?
Were there any disagreements or conflicts?
If so, what were the sources of conflict, and how might they be resolved?
\\
\hline
3 &
What seem to be the most popular opinions?
Do you agree with them?
Has your opinion changed over the course of the discussion?
\\
\hline
\end{tabular}
\end{table}

\subsection{Power Analysis}

\subsection{Quantifying Preferences, Conflict, and Agreement}

\section{Study Software}

\section{Pilot Studies}