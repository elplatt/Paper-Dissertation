\epigraph
{Pulling together is the aim of despotism and tyranny. Free men pull in all kinds of directions.}
{---Lord Vetinari in {\em The Truth}, Terry Pratchett}

The ability of large groups to reach mutually agreeable decisions is key to
democratic governance, organizational effectiveness, social movements, and peer production.
Faced with the intractability of large-scale participatory decision-making, traditional
systems have sacrificed one or more desirable properties, such as participation
(as in representative democracy), deliberation (as in voting), equality
(as in command hierarchies), and speed (as in formal consensus).
This dissertation examines the emerging role of internet-enabled collaborative
networks in overcoming these historical limitations.

In recent years, examples of large-scale collaborations have emerged that seem
to achieve the previously unachievable. Millions of volunteer Wikipedia editors
have created a high-quality encyclopedia without formal centralized leadership
\cite{keegan_evolution_2017, giles_internet_2005}.
The Free Software movement has produced the Linux kernel and GNU operating
system, which power much of the modern internet
\cite{coleman_coding_2012, benkler_coases_2002, raymond_cathedral_1999}.
Social movements such as the Arab Spring, Occupy, Black Lives Matter, and
Podemos have brought attention to deeply entrenched social issues,
and in some cases, contriubted to political regime changes
\cite{tufekci_twitter_2017, gonzalez-bailon_networked_2016}.
Participatory governance at such scales is unprecedented.
The emergence of these large-scale decentralized collaborations has been
attributed to the fast, bidirectional,
and global communication enabled by the internet
\cite{tufekci_twitter_2017, benkler_coases_2002}.
A better understanding of how specifically such communication sidesteps
historical barriers to large-scale collaboration will contribute to more
effective policy as well as best practices for organizational design and
intervention.
This dissertation focuses on one particularly challenging aspect of
such collaborations: decision-making.
Specifically, I examine how
when groups are too large for all members to participate in all discussions,
the course and outcome of the decision-making process is influenced by the
communication network structure: the shape of who talks to whom.

In this dissertation, I combine observational, agent-based modeling, and experimental
methods to evaluate the role of network structure in large-scale participatory decision-making.
I focus on group sizes well above the estimated 5--8 individual maximum for effective small goups
\cite{freeman_tyranny_1972, lohman_designing_2000, miflin_small_2004},
ranging from 32--33 participants in the experimental study to millions of editors in the
observational study of English-language Wikipedia WikiProjects.
Taken together, the findings in these studies suggests that small, tightly-knit pods,
interlocked through the sharing of members can benefit such collaborations.
This common structure, which I refer to as network deliberation, appears to preserve some
benefits of small-scale deliberation, while allowing efficient diffusion of information
through large groups.
Furthermore, I find evidence that network deliberation can mitigate some of the negative
aspects of social influence.
These findings provide insight into the success of existing large-scale collaborations
as well as provide guidance for the design of new sociotechnical systems.


\section{Theoretical Framework}

This dissertation draws on ideas from network science, economics, and complex
systems.
While a topic as broad as decision-making can be studied from many perspectives,
these fields provide a minimal framework for studying how individual preferences
and behaviors interact with interpersonal communication networks to influence
group decisions.

The fundamental challenge of large-scale collective decision-making is how to
reconcile the conflicting preferences of individual group members.
This challenge has been studied formally in social choice theory,
a sub-field of economics.
Prior work in social choice theory has found, somewhat discouragingly,
that even when all individual preferences are known perfectly,
making a fair collective decision isn't always possible.
In some cases, the method of aggregating individual preferences
(i.e. the voting system) can influence the outcome.
Social choice theory focuses on understanding of these limitations,
such as the Condorcet Paradox
\cite{condorcet_essay_1785} and
Arrow's Impossibility Theorem \cite{arrow_social_2012}.
While social choice theory typically assumes fixed fundamental preferences,
preferences can be instrumental in nature,
varying with factors such as available knowledge, group identity, or perspective.
This dissertation finds hope in the transformative potential of the
deliberative process.
Social choice theory typically assumes fixed individual preferences.
Deliberation allows individuals to influence and change each other's
preferences,
which creates the potential to sidestep the historical limitations of
social choice theory.
When individual preferences are allowed to vary, it becomes possible for an
irreconcilable set of preferences to evolve into one with a clear winner.
So far, this possibility has received relatively little attention,
most likely due to the historical intractability of large-scale deliberative
decision-making.
This dissertation explores the potential of effective large-scale deliberation, as enabled by the internet.
Such large-scale deliberation creates the potential for members to
resolve conflicting preferences and reach mutually acceptable decisions
without relying on coercive or hierarchical processes that might introduce
power imbalances or informational biases.

Network science provides the tools for analyzing the structure of interpersonal
networks.
Interpersonal interactions in large collaborations are necessarily structured:
when a group is too large for each individual to interact with all other individuals,
the question of ``who talks to whom?'' creates a network structure.
By modelling collaborative groups as a collection of abstract ``nodes''
connected by interpersonal communication links,
a group's communication structure can be studied in isolation.
Findings from network science suggest that social processes on networks can be
influenced by structural properties such as
the {\em degree}: the number of links a node has,
{\em geodesic distance} the number of links separating two nodes,
and {\em clustering}: how common it is for two linked nodes to share links with
a third \cite{boccaletti_complex_2006}.
While network structure is certainly not the only factor to influence collective
decision-making,
studying network structure in isolation provides a baseline for the
further study of social dynamics and other non-structural factors.
Network structure is also significant as a potential point of intervention in
cases where social dynamics may be difficult to influence.

This dissertation also incorporates theoretical and computational models from
social learning theory.
Social learning theory acknowledges that individuals rarely
learn or make decisions in isolation, but rather learn from and imitate others
in their social network
\cite{golub_naive_2010}.
Social learning theory
formalizes both the types of tasks collectives perform
\cite{hong_interpreted_2009}
and the behavioral strategies individuals might employ
\cite{lazer_network_2007, barkoczi_social_2016}.
These strategies range from pure imitation to critical evaluation,
depending on the circumstances being modeled.
Social learning models provide a baseline to compare empirical observations
against,
as well as a language and framework for placing findings into the context of the greater social learning literature.

Throughout this dissertation,
I motivate and develop a novel theoretical framework,
which I call {\em network deliberation}.
Network deliberation is an analytical framework that can be applied both to the empirical observation of existing sociotechical systems,
as well as to guide the creation and analysis of simplified models.
My review of the literature identifies a common theme
among successful large-scale internet-enabled collaborations:
large collectives composed of interlocking smaller groups.
These groups have various names, including:
committees, working groups, teams, circles, cores, syndicates,
affinity groups, zones, and nodes.
As an abstraction of these small interlocking group, I will use the term ``pods.''
Network deliberation describes large-scale collective deliberation achieved
through interlocking pods.
As in the theories of interlocking directorates \cite{levine_study_1979},
interlocking publics \cite{habermas_structural_1991},
and network rotation \cite{salehi_hive_2018},
pods allow for beneficial small group dynamics,
while the overlap between pods enables diffusion of information and opinions
through the greater collective.
The network deliberation framework abstracts and generalizes phenomena
such as interlocking directorates and interlocking publics,
allowing their commonalities to be studied from the perspective of network structure.
The framework also provides concrete parameters such as pod size and pod assignment method,
which can be used in quantitative analysis of models and experiments,
and provides a common language for making comparisons across models and experiments.
Network deliberation is conceptually similar to network rotation in that both describe
large-scale deliberation through repeated small-group interactions.
There are two key differences, one conceptual, and one relating to network structure.
Conceptually, network rotation reassigns pod membership one individual at a time in order
to optimize certain network properties,
while network deliberation focuses on providing an analytical framework for understanding
the impact of different pod assignment methods on network structure and deliberation outcome.
The one-at-a-time reassignments of network rotation also creates pods with high overlap,
while network deliberation is focused on simultaneous or sequential pods with little overlap.

In network deliberation, the method of assigning individuals to pods
(whether deliberate or self-organized) can produce interpersonal networks with
varying structures.
The central questions of this dissertation are 1. does an interlocking-pod structure improve deliberation relative to conventional structures? and 2. how does the structure of these interlocking-pod networks influence the process and outcome of deliberation
in large collaborations?

\section{Methodology}

Studying collective behavior on the scale of hundreds, thousands, and millions
presents significant methodological challenges.
To address these challenges, this dissertation combines multiple methodologies,
including: observational study, agent-based modelling, and field experiments.
I use observational studies to reconstruct real-world collaborative networks
from the English-language Wikipedia and analyze the collaborative output of
those networks (Chapter \ref{chap:wp-prod-perf}).
Observational study has the benefits of scaling to millions of individuals
in a real-world environment.
However, observational studies typically cannot establish causal relationships,
only correlation.

To begin to address the causal relationship between network structure and
deliberative outcome, I use agent-based models
(Chatpers \ref{chap:wp-prod-perf} and \ref{chap:abm}).
Agent-based models are computational models of large systems composed of
many agents following simple behavioral rules.
In this case, agents represent individual collaborators,
and their behavior is determined by their preferences and their strategy for
incorporating information learned from their neighbors.
Agent-based models can establish causality,
and do so in group sizes limited only by available computing power.
As simplified models, however, their results cannot necessarily be generalized
to real-world scenarios.

To bridge the limitations of the above methods,
this dissertation includes a controlled field experiment evaluating the
effect of network deliberation in real-world collective decision-making
(Chapter \ref{chap:experiment}).

\section{Contributions}
This dissertation describes the contributions of three projects.
Chapter \ref{chap:wp-prod-perf} describes an observational and computational
study of WikiProjects on the English-language Wikipedia.
This study examines one of the real-world collaborations that inspired the network deliberation
framework and shows that variations in the network structure of such a collaboration can
correlate with quality indicators.
Specifically, Chapter \ref{chap:wp-prod-perf} reports the following contributions:
\begin{itemize}
\setlength\itemsep{0pt}
\item Despite an overall productivity/performance trade-off,
more tightly-knit WikiProjects tend to produce articles more quickly as well as produce higher-quality articles;
\item The high performance of tightly-knit networks is consistent with individuals showing susceptibility to social influence;
\item Unequal participation is associated with lower performance;
\item The agent-based model shows that the relationship between quality and productivity can be influenced by both network structure and individual behavior.
\end{itemize}

Chapter \ref{chap:abm} describes an agent-based model of network deliberation,
comparing performance across several network topologies and social learning
strategies.
The primary contributions are:
\begin{itemize}
\setlength\itemsep{0pt}
\item When agents are strongly susceptible to social influence,
Network deliberation identifies solutions of higher quality than
conventional deliberation,
while requiring less time to converge.
\item Within network deliberation,
agents who base behavior on social influence peroform best on networks with short paths.
However, when agents act on their own judgement,
they perform best on networks with longer paths,
consistent with findings for conventional deliberation.
\cite{barkoczi_social_2016}.
\item A novel social learning strategy, confident-neighbor,
which outperforms conventional strategies across all
networks despite relying on strictly less information.
\end{itemize}

Chapter \ref{chap:experiment} describes an experimental study network deliberation in large-scale human collaborations.
The study design uses periodic ranked-choice polls to track individual preferences
throughout the course of a large-scale online deliberation.
By varying communication network structure and tracking the evolution of individual
preferences,
this experiment evaluates the ability of network deliberation to resolve conflict
and build consensus, relative to conventional single-group deliberation.
The primary contributions are:
\begin{itemize}
\setlength\itemsep{0pt}
    \item Findings support the hypothesis that network deliberation is better at facilitating agreement than conventional deliberation.
    \item Findings suggest that network deliberation can provide protection against information cascades.
    \item Findings include no evidence that network deliberation facilitates substantial conflict-resolution, but indicate that it may provide protection against polarization.
\end{itemize}

The above contributions suggest that network deliberation structure may be an important factor in the
success of existing large-scale, internet-enabled collaborations, and furthermore,
that it may help guide the design and organization of future collaborations.
Specifically, network deliberation appears to help moderate the effects of social influence,
providing some protection against both information cascades due to positive social influence,
and polarization due to negative social influence.
In summary, the work presented in the following chapters provides a framework for the
future study of large-scale participatory governance,
and provides evidence that large-scale participatory governance can be both practical and effective.