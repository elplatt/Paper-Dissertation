As groups grow in size, they gain access to additional resources, creating opportunities for collective intelligence and collective action.
However, at very large scales, group decision-making becomes prohibitively slow and difficult to coordinate.
Traditional solutions include representative decision-making and/or a shift from deliberation to voting.
Both approaches sacrifice desirable properties.
Representative decision-making loses the potential benefits of collective intelligence and introduces hierarchies that may place the interests of specific individuals ahead of the interests of the group.
Voting sacrifices generativity: allowing a choice between predefined options, without allowing for improvement to those options.
Arrow’s impossibility theorem also fundamentally limits the fairness of voting.
This project proposes Networked Deliberation as a potential means of effective large-scale decision-making.
In Networked Deliberation, members of a large group are repeatedly partitioned into small deliberative pods.
Overlap between pods at different stages enables group-wide diffusion of information and preferences.
Different methods for assigning members to pods result in different network topologies.
This work combines observational study, agent-based modeling, and an experiment to evaluate and better understand network deliberation.
An empirical observation of WikiProjects on the English-language Wikipedia identifies the network properties of the most effective projects.
A simple agent-based model of Network Deliberation shows improvements over conventional deliberation in the presence of strong social influence.
Finally, a controlled experiment studies network deliberation in a real world setting, tracking how individual preferences evolve, and finding evidence that network deliberation provides protection against negative consequences of social influence.
This work seeks to provide a framework that enables an understanding of
how very large groups can quickly develop a consensus, enabling a more effective use of shared resources to achieve common goals.