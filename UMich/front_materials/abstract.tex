As groups grow in size, they gain access to additional resources,
creating opportunities for collective intelligence and collective action.
However, at very large scales, group decision-making becomes prohibitively slow and difficult
to coordinate.
Traditional solutions include representative decision-making and/or a shift from deliberation
to voting.
Both approaches sacrifice desirable properties.
Representative decision-making loses the potential benefits of collective intelligence and
introduces hierarchies that may place the interests of specific individuals ahead of the
interests of the group.
Voting sacrifices generativity: allowing a choice between predefined options,
without allowing for improvement to those options.
Arrow’s impossibility theorem also fundamentally limits the fairness of voting.
This project proposes Networked Deliberation as a potential means of large-scale decision-making.
In Networked Deliberation, members of a large group are repeatedly partitioned into small
deliberative pods.
Overlap between pods at different stages enables group-wide diffusion of information and
preferences.
Different methods for assigning members to pods result in different network topologies.
In partnership with large community groups, this project will use a web-based platform to
evaluate Network Deliberation.
For a variety of network topologies, the speed and quality of preference convergence will be
evaluated using quantitative information from ranked-choice polls throughout the process.
Deliberator satisfaction and experience will also be evaluated using qualitative analysis of
deliberative text and surveys.
This work seeks to improve the ability of very large groups to quickly develop a consensus,
enabling a more effective use of shared resources to achieve common goals.