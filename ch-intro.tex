The internet has enabled collaborations at a scale never before possible,
but the best practices for organizing such large collaborations are still not clear.
Wikipedia is a visible and successful example of such a collaboration which might offer
insight into what makes large-scale, decentralized collaborations successful.
We analyze the relationship between the structural properties of WikiProject coeditor networks
and the performance and efficiency of those projects.
We confirm the existence of an overall performance-efficiency trade-off,
while observing that some projects are higher than others in both performance
and efficiency,
suggesting the existence factors correlating positively with both.
Namely, we find an association between low-degree coeditor networks
and both high performance and high efficiency.
We also confirm results seen in previous numerical and small-scale lab studies:
higher performance with less skewed node distributions,
and higher performance with shorter path lengths.
We use agent-based models to explore possible mechanisms for
degree-dependent performance and efficiency.
We present a novel local-majority learning strategy designed to satisfy properties
of real-world collaborations.
The local-majority strategy as well as a localized conformity-based strategy
both show degree-dependent performance and efficiency,
but in opposite directions,
suggesting that these factors depend on both network structure and learning strategy.
Our results suggest 
possible benefits to decentralized collaborations made of smaller,
more tightly-knit teams,
and that these benefits may be modulated by the particular learning strategies
in use.

Deliberation, a form of collective problem-solving, is a key component of
democracy, social movements, and online peer-production.
The network structure of interpersonal communication plays an important role in
collective problem-solving, particularly in deliberation, which can become
prohibitively complex and time-intensive at large scales.
Network properties such as structural efficiency and degree distribution have
been shown to influence the speed and quality of collective problem-solving,
in combination with the behavioral strategies employed by individuals.
However, many successful examples of mass deliberation differ in structure from
common models of collective problem-solving by exhibiting networks of small
interlocking groups.
In this paper, we present an agent-based model of deliberation in networks of
small interlocking groups and compare performance on these networks with
conventional network structures.
We find that networks of small interlocking groups improve solution quality
when problem-solvers exhibit strong social influence.
This effect is compatible with, but distinct from, the previously observed
benefit of network efficiency in the presence of strong social influence.
Our findings suggest a possible mechanism contributing to the success of
existing mass deliberative projects as well as a principle for the design of
new projects.
By contributing to more effective deliberation at larger scales,
we hope this work will contribute to democratizing the governance of large
sociotechnical systems.
