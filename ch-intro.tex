The internet has enabled collaborations at a scale never before possible,
but the best practices for organizing such large collaborations are still not clear.
Wikipedia is a visible and successful example of such a collaboration which might offer
insight into what makes large-scale, decentralized collaborations successful.
We analyze the relationship between the structural properties of WikiProject coeditor networks
and the performance and efficiency of those projects.
We confirm the existence of an overall performance-efficiency trade-off,
while observing that some projects are higher than others in both performance
and efficiency,
suggesting the existence factors correlating positively with both.
Namely, we find an association between low-degree coeditor networks
and both high performance and high efficiency.
We also confirm results seen in previous numerical and small-scale lab studies:
higher performance with less skewed node distributions,
and higher performance with shorter path lengths.
We use agent-based models to explore possible mechanisms for
degree-dependent performance and efficiency.
We present a novel local-majority learning strategy designed to satisfy properties
of real-world collaborations.
The local-majority strategy as well as a localized conformity-based strategy
both show degree-dependent performance and efficiency,
but in opposite directions,
suggesting that these factors depend on both network structure and learning strategy.
Our results suggest 
possible benefits to decentralized collaborations made of smaller,
more tightly-knit teams,
and that these benefits may be modulated by the particular learning strategies
in use.
