The ability of large groups to reach mutually agreeable decisions is key to
democratic governance, social movements, and peer production (e.g., Wikipedia).
Faced with the intractability of large-scale decision-making, traditional
systems have sacrificed one or more desirable properties, such as participation
(as in representative democracy), deliberation (as in voting), equality
(as in command hierarchies), and speed (as in formal consensus).
This dissertation examines the emerging role of internet-enabled collaborative
networks in overcoming these historical limitations.

In recent years, examples of large-scale collaborations have emerged that seem
to achieve the previously unachievable. Millions of volunteer Wikipedia editors
have created a high-quality encyclopedia without centralized leadership
\cite{keegan_evolution_2017, giles_internet_2005}.
The Free Software movement has produced the Linux kernel and GNU operating
system, which power much of the modern internet
\cite{coleman_coding_2013, benkler_coases_2002, raymond_cathedral_1999}.
Social movements such as the Arab Spring, Occupy, Black Lives Matter, and
Podemos have reshaped national governments and brought attention to deeply
entrenched social issues
\cite{tufekci_twitter_2017, gonzalez-bailon_networked_2016}.
The emergence of these large-scale decentralized collaborations has been
attributed to the fast, bidirectional,
and global communication enabled by the internet
\cite{tufekci_twitter_2017, benkler_coases_2002}.
A better understanding of how specifically such communication sidesteps
historical barriers to large-scale collaboration will contribute to more
effective policy as well as best practices for organizational design and
intervention.
This dissertation focuses on one particularly challenging aspect of
such collaborations: decision-making
I specificaly examine how,
when groups are too large for all members to participate in all discussions,
the course and outcome of the decision-making process is influenced by the
communication network structure: the shape of who talks to whom.

\section{Theoretical Framework}

This dissertation draws on ideas from network science, economics, and complex
systems.
While a topic as broad as decision-making can be studied from many perspectives,
these fields provide a minimal framework for studying how individual preferences
and behaviors interact with interpersonal communication networks to influence
group decisions.

The fundamental challenge of large-scale collective decision-making is how to
reconcile the conflicting preferences of individual group members.
This challenge has been studied formally in social choice theory,
a sub-field of economics.
Prior work in social choice theory has found, somewhat discouragingly,
that even when all individual preferences are known perfectly,
making a fair collective decision isn't always possible.
In some cases, the method of aggregating individual preferences
(i.e. the voting system) can influence the outcome.
Social choice theory focuses on understanding of these limitations,
such as the Condorcet Paradox
\cite{condorcet_essay_1785} and
Arrow's Impossibility Theorem \cite{arrow_social_2012}.
This dissertation finds hope in the transformative potential of the
deliberative process.
Social choice theory typically assumes fixed individual preferences.
Deliberation allows individuals to influence and change each other's
preferences,
which creates the potential to sidestep the historical limitations of
social choice theory.
When individual preferences are allowed to vary, it becomes possible for an
irreconcilable set of preferences to evolve into one with a clear winner.
So far, this possibility has received relatively little attention,
most likely due to the historical intractability of large-scale deliberative
decision-making.
This dissertation explores the potential of the internet to enable effective
deliberation in large collectives.
Such large-scale deliberation creates the potential for members to
resolve conflicting preferences and reach mutually acceptable decisions
without relying on coercive or hierarchical processes that might introduce
power imbalances or informational biases.

Network science provides the tools for analyzing the structure of interpersonal
networks.
Interpersonal interactions in large collaborations are necessarily structured:
when a group is too large for each individual to interact with all other individuals,
the question of ``who talks to whom?'' creates a network structure.
By modelling collaborative groups as a collection of abstract ``nodes''
connected by interpersonal communication links,
a group's communication structure can be studied in isolation.
Findings from network science suggest that social processes on networks can be
influenced by structural properties such as
the {\em degree}: the number of links a node has,
{\em geodesic distance} the number of links separating two nodes,
and {\em clustering}: how common it is for two linked nodes to share links with
a third \cite{boccaletti_complex_2006}.
While network structure is certainly not the only factor to influence collective
decision-making,
studying network structure in isolation provides a baseline for the
further study of social dynamics and other non-structural factors.
Network structure is also significant as a potential point of intervention in
cases where social dynamics may be difficult to influence.

This dissertation also incorporates theoretical and computational models from
social learning theory.
Social learning theory acknowledges that individuals rarely
learn or make decisions in isolation, but rather learn from and imitate others
in their social network
\cite{golub_naive_2010}.
Social learning theory
formalizes both the types of tasks collectives perform
\cite{hong_interpreted_2009}
and the behavioral strategies individuals might employ
\cite{lazer_network_2007, barkoczi_social_2016}.
These strategies range from pure imitation to critical evaluation,
depending on the circumstances being modeled.
Social learning models provide a baseline to compare empirical observations
against,
as well as a language and framework for contextualizing findings.

Throughout this dissertation,
I motivate and develop a novel theoretical framework,
which I call {\em network deliberation}.
My review of the literature identifies a common theme
among successful large-scale internet-enabled collaborations:
large collectives composed of interlocking smaller groups.
These groups have various names, including:
committees, working groups, teams, circles, cores, syndicates,
affinity groups, zones, and nodes.
As an abstraction of these small interlocking group, I will use the term "pods."
Network deliberation describes large-scale collective deliberation achieved
through interlocking pods.
As in the theories of interlocking directorates \cite{levine_study_1979},
interlocking publics \cite{habermas_structural_1991},
and network rotation \cite{salehi_hive_2018},
pods allow for beneficial small group dynamics,
while the overlap between pods enables diffusion of information and opinions
through the greater collective.
In network deliberation, the method of assigning individuals to pods
(whether deliberate or self-organized) can produce interpersonal networks with
varying structures.
The central question of this dissertation is how the structure of these
interlocking-pod networks influences the process and outcome of deliberation
in large collaborations.

\section{Methodology}

Studying collective behavior on the scale of hundreds, thousands, and millions
presents significant methodological challenges.
To address these challenges, this dissertation combines multiple methodologies,
including: observational study, agent-based modelling, and field experiment.
I use observational studies to reconstruct real-world collaborative networks
from the English-language Wikipedia and analyze the collaborative output of
those networks (Chapter \ref{chap:wp-prod-perf}).
Observational study has the benefits of scaling to millions of individuals
in a real-world environment.
However, observational studies typically cannot establish causal relationships,
only correlation.

To begin to address the causal relationship between network structure and
deliberative outcome, I use agent-based models
(Chatpers \ref{chap:wp-prod-perf} and \ref{chap:abm}).
Agent-based models are computational models of large systems composed of
many agents following simple behavioral rules.
In this case, agents represent individual collaborators,
and their behavior is determined by their preferences and their strategy for
incorporating information learned from their neighbors.
Agent-based models can establish causality,
and do so in group sizes limited only by available computing power.
As simplified models, however, their results cannot necessarily be generalized
to real-world scenarios.

To bridge the limitations of the above methods,
this dissertation will include a controlled field experiment evaluating the
effect of network deliberation in real-world collective decision-making
(Chapter \ref{chap:experiment}).

\section{Contributions}
This dissertation describes the contributions of three projects.
Chapter \ref{chap:wp-prod-perf} describes an observational and computational
study of WikiProjects on the English-language Wikipedia,
and reports the following contributions:
\begin{itemize}
\setlength\itemsep{0pt}
\item Despite an overall productivity/performance trade-off,
WikiProjects with low-degree coeditor networks tend to have both higher
productivity and higher performance;
\item Short geodesic lengths are associated with higher performance,
consistent with a conformity-based learning strategy;
\item Structural inequality, as measured by degree skewness,
is associated with lower performance;
\item The agent-based model shows that the productivity and performance of
collaborations can depend on network degree,
and that the direction of that dependence can depend on social learning strategy.
\end{itemize}

Chapter \ref{chap:abm} describes an agent-based model of network deliberation,
comparing performance across several network topologies and social learning
strategies.
The primary contributions are:
\begin{itemize}
\setlength\itemsep{0pt}
\item When agents conform to social influence,
Network deliberation identifies solutions of higher quality than
conventional deliberation,
while requiring less time to converge.
\item Within network deliberation conditions,
the structurally-efficient random pod network outperforms the
structurally-inefficient long-path network when agents conform to social influence.
However, when agents prefer their own judgement to social influence,
the inefficient network yields higher performance,
consistent with findings for conventional networks.
\cite{barkoczi_social_2016}.
\item A novel social learning strategy, confident-neighbor,
which outperforms or matches the conventional best-neighbor strategy across all
networks considered,
despite using strictly less information about solution quality.
\end{itemize}

Chapter \ref{chap:experiment} proposes and describes current progress on a
field experiment to study network deliberation in large-scale human collaborations.
The study design uses periodic ranked-choice polls to track individual preferences
throughout the course of a large-scale online deliberation.
By varying communication network structure and tracking the evolution of individual
preferences,
this experiment evaluate the ability of network deliberation to resolve conflict
and build consensus, relative to conventional single-group deliberation.

\section{Dissertation Progress and Timeline}
At the time of this proposal,
the analysis of English-language WikiProjects described in Chapter
\ref{chap:wp-prod-perf} has been completed and resulted in a published
paper \cite{platt_network_2018}.
The agent-based modelling project described in Chapter \ref{chap:abm}
has been completed and resulted in a finished manuscript.
I have completed custom software, obtained IRB exemption, and conducted
several pilot studies for the field experiment described in
Chapter \ref{chap:experiment}.
I expect to have data collected from at least one full-scale experiment
by the end of 2021, and to have the data analyzed by mid-February, 2022.
I plan to have my dissertation completed by mid-May, 2022.
