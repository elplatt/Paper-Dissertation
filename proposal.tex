\documentclass{report}

\usepackage{apacite}
\usepackage{bbding}
\usepackage{courier}
\usepackage{epigraph}
\usepackage{graphicx}
\usepackage{helvet}
\usepackage[normal]{threeparttable}
\usepackage{natbib}
\usepackage{times}
\usepackage{titlesec}
\usepackage{url}

\newcommand{\beq}{\begin{eqnarray}}
\newcommand{\eeq}{\end{eqnarray}}
\newcommand{\+}{\phantom{-}}

\author{Edward L. Platt}
\title{Network Deliberation: The role of network structure in large-scale, internet-enabled, participatory decision-making}

\begin{document}

\maketitle

\begin{abstract}
As groups grow in size, they gain access to additional resources,
creating opportunities for collective intelligence and collective action.
However, at very large scales, group decision-making becomes prohibitively slow and difficult
to coordinate.
Traditional solutions include representative decision-making and/or a shift from deliberation
to voting.
Both approaches sacrifice desirable properties.
Representative decision-making loses the potential benefits of collective intelligence and
introduces hierarchies that may place the interests of specific individuals ahead of the
interests of the group.
Voting sacrifices generativity: allowing a choice between predefined options,
without allowing for improvement to those options.
Arrow’s impossibility theorem also fundamentally limits the fairness of voting.
This project proposes Networked Deliberation as a potential means of large-scale decision-making.
In Networked Deliberation, members of a large group are repeatedly partitioned into small
deliberative pods.
Overlap between pods at different stages enables group-wide diffusion of information and
preferences.
Different methods for assigning members to pods result in different network topologies.
In partnership with large community groups, this project will use a web-based platform to
evaluate Network Deliberation.
For a variety of network topologies, the speed and quality of preference convergence will be
evaluated using quantitative information from ranked-choice polls throughout the process.
Deliberator satisfaction and experience will also be evaluated using qualitative analysis of
deliberative text and surveys.
This work seeks to improve the ability of very large groups to quickly develop a consensus,
enabling a more effective use of shared resources to achieve common goals.
\end{abstract}

\tableofcontents

\chapter{Introduction}
The internet has enabled collaborations at a scale never before possible,
but the best practices for organizing such large collaborations are still not clear.
Wikipedia is a visible and successful example of such a collaboration which might offer
insight into what makes large-scale, decentralized collaborations successful.
We analyze the relationship between the structural properties of WikiProject coeditor networks
and the performance and efficiency of those projects.
We confirm the existence of an overall performance-efficiency trade-off,
while observing that some projects are higher than others in both performance
and efficiency,
suggesting the existence factors correlating positively with both.
Namely, we find an association between low-degree coeditor networks
and both high performance and high efficiency.
We also confirm results seen in previous numerical and small-scale lab studies:
higher performance with less skewed node distributions,
and higher performance with shorter path lengths.
We use agent-based models to explore possible mechanisms for
degree-dependent performance and efficiency.
We present a novel local-majority learning strategy designed to satisfy properties
of real-world collaborations.
The local-majority strategy as well as a localized conformity-based strategy
both show degree-dependent performance and efficiency,
but in opposite directions,
suggesting that these factors depend on both network structure and learning strategy.
Our results suggest 
possible benefits to decentralized collaborations made of smaller,
more tightly-knit teams,
and that these benefits may be modulated by the particular learning strategies
in use.

Deliberation, a form of collective problem-solving, is a key component of
democracy, social movements, and online peer-production.
The network structure of interpersonal communication plays an important role in
collective problem-solving, particularly in deliberation, which can become
prohibitively complex and time-intensive at large scales.
Network properties such as structural efficiency and degree distribution have
been shown to influence the speed and quality of collective problem-solving,
in combination with the behavioral strategies employed by individuals.
However, many successful examples of mass deliberation differ in structure from
common models of collective problem-solving by exhibiting networks of small
interlocking groups.
In this paper, we present an agent-based model of deliberation in networks of
small interlocking groups and compare performance on these networks with
conventional network structures.
We find that networks of small interlocking groups improve solution quality
when problem-solvers exhibit strong social influence.
This effect is compatible with, but distinct from, the previously observed
benefit of network efficiency in the presence of strong social influence.
Our findings suggest a possible mechanism contributing to the success of
existing mass deliberative projects as well as a principle for the design of
new projects.
By contributing to more effective deliberation at larger scales,
we hope this work will contribute to democratizing the governance of large
sociotechnical systems.


\chapter{Network structure, productivity, and performance in WikiProjects}
\include{ch-wp-prod-perf.tex}

\chapter{Small interlocking groups improve mass deliberation in the presence
of strong social influence}

\chapter{Experimental evaluation of Network Deliberaiton}

\chapter*{Acknowledgements}
Thanks to Daniel M. Romero.


Scott E. Page,
Tawanna Dillahunt,
Ceren Budak,
for their time, guidance, and helpful feedback.

Jane Im,

Danielle Livneh,
Karthik Ramanathan
for help collecting WikiProjects data.

Yan Chen,
Tanya Rosenblat,

Grant Schoenbeck,
Kentaro Toyama,
Charles R. Severance

Veronica Falandino

Jeffrey W. Lockhart,
Danaja Maldeniya,
Lia Bozarth,
Christopher Quarles,
Chanda Phelan,
Padma Chirumamilla,
Earnest Wheeler,
Sam Carton,
Elliott Brannon,
Tawfiq Ammari,
2015 UMSI PhD chohort
Jean Hardy,
Cindy Lin,

the attendees of the May 25, 2017 MIT Center for Civic Media lab meeting
the Berkman-Klein Center Cooperation Working Group
the 2019 CSCW Workshop on Team and Group Diversity
for helpful feedback;

This research was partly supported by
the National Science Foundation under Grant No. IIS-1617820.

Andy Brosius,
Persephone Hernandez-Vogt,


\nocite{*}
\bibliographystyle{apacite}
\bibliography{prelim}

\end{document}
