\documentclass{sig-alternate-05-2015}

%\usepackage[ruled]{algorithm2e}
%\renewcommand{\algorithmcfname}{ALGORITHM}
%\usepackage{amsmath,amsfonts,amssymb,bbm} 
%\usepackage[numbers,sort&compress]{natbib} % for citet
%\SetArgSty{textrm}  % for algorithm2e
%\SetAlFnt{\small}
%\SetAlCapFnt{\small}
%\SetAlCapNameFnt{\small}
%\SetAlCapHSkip{0pt}

\usepackage[utf8]{inputenc}
\usepackage{tikz}
\usetikzlibrary{arrows,chains,decorations.pathreplacing}
%\usepackage{draftwatermark}
%\SetWatermarkText{DRAFT}
\newcommand{\beq}{\begin{eqnarray}}
\newcommand{\eeq}{\end{eqnarray}}
\DeclareMathOperator{\wbf}{wBF}

\begin{document}

% Copyright
\setcopyright{acmcopyright}
%\setcopyright{acmlicensed}
%\setcopyright{rightsretained}
%\setcopyright{usgov}
%\setcopyright{usgovmixed}
%\setcopyright{cagov}
%\setcopyright{cagovmixed}


% DOI
%\doi{10.475/123_4}

% ISBN
%\isbn{123-4567-24-567/08/06}

%Conference
\conferenceinfo{WSDM '17}{February 6--10, 2017, Cambridge, UK, USA}

\acmPrice{\$15.00}

%
% --- Author Metadata here ---
%\conferenceinfo{WOODSTOCK}{'97 El Paso, Texas USA}
%\CopyrightYear{2007} % Allows default copyright year (20XX) to be over-ridden - IF NEED BE.
%\crdata{0-12345-67-8/90/01}  % Allows default copyright data (0-89791-88-6/97/05) to be over-ridden - IF NEED BE.
% --- End of Author Metadata ---

\title{Attack-Tolerant Networks: A Structural, Multipath Approach}
%
% You need the command \numberofauthors to handle the 'placement
% and alignment' of the authors beneath the title.
%
% For aesthetic reasons, we recommend 'three authors at a time'
% i.e. three 'name/affiliation blocks' be placed beneath the title.
%
% NOTE: You are NOT restricted in how many 'rows' of
% "name/affiliations" may appear. We just ask that you restrict
% the number of 'columns' to three.
%
% Because of the available 'opening page real-estate'
% we ask you to refrain from putting more than six authors
% (two rows with three columns) beneath the article title.
% More than six makes the first-page appear very cluttered indeed.
%
% Use the \alignauthor commands to handle the names
% and affiliations for an 'aesthetic maximum' of six authors.
% Add names, affiliations, addresses for
% the seventh etc. author(s) as the argument for the
% \additionalauthors command.
% These 'additional authors' will be output/set for you
% without further effort on your part as the last section in
% the body of your article BEFORE References or any Appendices.

\numberofauthors{1} %  in this sample file, there are a *total*
% of EIGHT authors. SIX appear on the 'first-page' (for formatting
% reasons) and the remaining two appear in the \additionalauthors section.
%
\author{
% You can go ahead and credit any number of authors here,
% e.g. one 'row of three' or two rows (consisting of one row of three
% and a second row of one, two or three).
%
% The command \alignauthor (no curly braces needed) should
% precede each author name, affiliation/snail-mail address and
% e-mail address. Additionally, tag each line of
% affiliation/address with \affaddr, and tag the
% e-mail address with \email.
%
% 1st. author
\alignauthor
Anonymized Authors
}
%       \affaddr{Institute for Clarity in Documentation}\\
%       \affaddr{1932 Wallamaloo Lane}\\
%       \affaddr{Wallamaloo, New Zealand}\\
%       \email{trovato@corporation.com}
% 2nd. author
%\alignauthor
%G.K.M. Tobin\titlenote{The secretary disavows
%any knowledge of this author's actions.}\\
%       \affaddr{Institute for Clarity in Documentation}\\
%       \affaddr{P.O. Box 1212}\\
%       \affaddr{Dublin, Ohio 43017-6221}\\
%       \email{webmaster@marysville-ohio.com}
% 3rd. author

\maketitle
\begin{abstract}
Networks with single points of failure are particularly susceptible to
adversarial faults, in which an attacker targets nodes strategically.
In communication networks, these types of attacks leave users vulnerable to
censorship and targeted surveillance.
Centralized networks have single points of failure by definition,
leading to a growing popularity in decentralized architectures and protocols.
However, centralized network structure can arise even when protocols are
decentralized.
While based on decentralized protocols, the Internet and World-Wide Web have
been shown both theoretically and historically to be highly susceptible to
adversarial faults, in part due to emergent, structural centralization.
Existing network trust models, such as webs of trust, fail to adequately address
network structure. We develop a partial trust model that incorporates network
structure and which makes weaker, more realistic transitivity assumptions than
webs of trust. We show that the partial trust model, when combined with
concurrent multipath routing, allows standard fault tolerance techniques to be
applied to communication across complex networks. We also provide a specific
routing algorithm for implementing this scheme on the butterfly network
topology.
When network topology can be dictated, these results can be used to create
scalable, attack-tolerant infrastructures. More generally, our results provide
a formalism for evaluating the effects of network structure on adversarial
fault tolerance.
\end{abstract}


%
% The code below should be generated by the tool at
% http://dl.acm.org/ccs.cfm
% Please copy and paste the code instead of the example below. 
%
\begin{CCSXML}
<ccs2012>
<concept>
<concept_id>10002978.10002986</concept_id>
<concept_desc>Security and privacy~Formal methods and theory of security</concept_desc>
<concept_significance>500</concept_significance>
</concept>
<concept>
<concept_id>10003033.10003034</concept_id>
<concept_desc>Networks~Network architectures</concept_desc>
<concept_significance>500</concept_significance>
</concept>
<concept>
<concept_id>10011007.10010940.10010971.10010972.10010540</concept_id>
<concept_desc>Software and its engineering~Peer-to-peer architectures</concept_desc>
<concept_significance>500</concept_significance>
</concept>
</ccs2012>
\end{CCSXML} 

\ccsdesc[500]{Security and privacy~Formal methods and theory of security}
\ccsdesc[500]{Networks~Network architectures}
\ccsdesc[500]{Software and its engineering~Peer-to-peer architectures}

%
% End generated code
%

%
%  Use this command to print the description
%
\printccsdesc

% We no longer use \terms command
%\terms{Theory}

\keywords{networks; decentralization; fault tolerance; routing; multipath routing; adversarial faults; information security}

\section{Introduction}

Large-scale communication networks, exemplified by the Internet,
have become ubiquitous, and are now crucial infrastructure for
individuals, communities, and organizations around the world.
As with any critical infrastructure, the cost of a failure can be
immense, so methods for tolerating various kind of faults are an
important and ongoing area of research
\cite{zin_survey_2015,albert_error_2000,sterbenz_resilience_2010}.
In fact, the decentralized design of the Internet was intended to protect
against targeted attacks, such as nuclear strikes
\cite{baran_distributed_1964}.
Such targeted attacks are formally described as {\em adversarial faults}
and are some of the most difficult to protect against.
Both theoretical results and recent events (described below)
have demonstrated that
the Internet is surprisingly vulnerable to adversarial faults.
Decentralization remains a promising approach to building resilient networks,
but further work is needed to understand the relationship between
decentralized network structure and adversairal fault tolerance.

Analysis of the Internet's router network has shown that while it is remarkably
resilient against random faults,
it is highly susceptible to adversarial faults \cite{albert_error_2000}.
These results have been attributed to the scale-free structure of the Internet's
router network
\cite{barabasi_emergence_1999,barabasi_scale-free_2009}.
In scale-free networks and other networks with heavy-tail degree distributions,
random failures are highly likely to pick low-degree nodes, thus having
little effect,
while adversarial faults target the small number of high-degree nodes,
thus removing a large number of edges with each fault.
It may seem to be a contradiction that the Internet both be decentralized
and exhibit a heavy-tailed degree distribution.
This contradiction is resolved by observing that the Internet is decentralized
in its {\em protocols}, but somewhat centralized in its {\em network structure}.
In other words, the protocols of the Internet do not {\em require}
centralization, but centralization may still emerge from the socio-technical
processes that create its network structure.

The Internet's vulnerability to censorship and other targeted attacks
has been confirmed by the success of several such attacks.
For example, in 2008, YouTube suffered a worldwide outage for several hours
when a service provider in Pakistan advertised false routing information
\cite{hunter_pakistan_2008}.
The action (known as a {\em black hole attack}) was intended to censor YouTube
within Pakistan only, but resulted in a worldwide cascading failure.
Such vulnerabilities are not limited to any one system or protocol,
but result from the network structure itself.
With strong theoretical and historical evidence that network structure
can create vulnerabilities,
methods for analyzing structural vulnerabilities and for designing
fault tolerant networks are needed.
This paper presents several contributions towards advancing those goals.

We consider a setting in which a source node (Alice)
in a network attempts to route a message
to a target node (Bob) by forwarding it through the links of the network.
We assume that some nodes in the network may be compromised by an attacker
(Mal).
We assume that Mal has full knowledge of the network structure, but has
limited resources and thus can only compromise a fixed number of nodes.
Compromised nodes may behave incorrectly by blocking the message,
changing the message content, or routing the message incorrectly.
In order to evaluate network-based techniques for mitigating such attacks,
we develop a decentralized {\em partial trust} model,
similar to the commonly used web of trust approach
\cite{zimmermann_official_1995,ferguson_practical_2003},
but making weaker (more realistic) assumptions of trust transitivity,
and incorporating network structure.
The partial trust model can be used to design and analyze communication
schemes that tolerate adversarial faults,
even when no single path through the network is fully trusted.

While {\em confidentiality} is a common concern in secure communication,
it is not the focus of this paper.
Existing cryptographic techniques for confidentiality are relatively
mature compared to those for tolerating censorship.
In fact, an attacker might use targeted attacks to achieve censorship
as a fallback when surveillance isn't feasible.
However, we stress that the technqiues presented in this paper
are entirely compatible with, and in some cases could enhance, existing
privacy and secrecy techniques.
For example, the well-known {\em man-in-the-middle} attack exploits a privileged
network position to attack otherwise secure cryptography,
suggesting that structural approaches can complement cryptographic ones.

Using the partial trust model, we are able to reduce complex networks to
simply a number of redundant, independent paths.
This reduction enables the application of standard error detection techniques,
which rely on statistically independent errors.
While adversarial faults along a single path are correlated,
faults in randomly-chosen independent paths are not.
Thus fault tolerance techniques can be applied,
even in the presence of adversarial errors,
when messages are sent in parallel,
a technique known as {\em concurrent multipath routing}
\cite{zin_survey_2015, qadir_exploiting_2015}.
The receiver can then use the redundant messages to detect and/or correct
errors.
Using the partial trust model, we formally evaluate the 
effects of network structure on attack-resistance and show that the probability
of undetected errors decreases exponentially with the number of
independent paths between source and destination,
even when no individual path is entirely trusted.

Our specific multipath fault tolerance scheme relies on the existence
and discoverability of multiple redundant paths between the source and
the destination.
Implementation thus requires a routing algorithm.
While not a requirement, this technique is well-suited for
{\em structured networks},
in which links between nodes are constrained to a particular architecture.

In particular, we provide a novel concurrent multipath routing algorithm
on the butterfly network topology and use the partial trust model to prove
a high level of adversarial fault tolerance.
The butterfly topology is popular in parallel processing
\cite{kshemkalyani_distributed_2008} and
peer-to-peer \cite{lua_survey_2005, korzun_structured_2013}
applications, due to its regular structure and high connectivity.
While the ability to control network topology is only achievable in some
applications, those applications are nonetheless significant,
e.g. overlay networks \cite{lua_survey_2005, korzun_structured_2013},
formal organizations \cite{mohr_explaining_1982},
government-regulated cellular networks \cite{walker_mass_2012},
and call tree notification systems \cite{nickerson_thinking_2010}.

Our main contributions are:
\begin{itemize}
\item{We develop a decentralized {\em partial trust model},
which makes weaker transitivity assumptions than web-of-trust models,
and makes it possible to analyze the effect of network structure on
adversarial fault tolerance;}
\item{We show that the partial trust model enables standard fault
tolerance techniques to be applied to adversarial faults in
complex communication networks.
We find that the probability of detecting adversarial faults
approaches 1 exponentially as the number of independent paths between
sender and receiver increases;}
\item{We present a scalable, efficient, and attack-tolerant concurrent
multipath routing algorithm on the butterfly network topology.}
\end{itemize}

This paper is organized as follows.
Section 2. reviews background and related work.
Section 3. discusses the desired properties of attack-tolerant network
infrastructure.
Section 4. describes the partial trust model.
Section 5. describes a general technique for multipath fault tolerance on
structured networks.
Section 6. describes a routing algorithm for multipath fault tolerance
on the butterfly network topology.
Section 7. discusses the our results.
And Section 8. concludes.

\section{Background and Related Work}

In {\em structured networks}, link structure is predetermined.
Such networks can be designed to have favorable structural and routing properties,
at the expense of complicating the addition or removal of nodes.
Structured networks have been a popular tool in parallel and distributed
computing architectures \cite{kshemkalyani_distributed_2008}.
More recently, peer-to-peer systems based on distributed hash tables have used
structured ``overlay'' networks to map table keys to local TCP/IP routes
\cite{lua_survey_2005,korzun_structured_2013}.

Many distributed consensus protocols (such as those used by crypto-currencies)
are designed to provide tolerance against arbitrary or adversarial faults.
Byzantine agreement protocols
\cite{lamport_byzantine_1982,castro_practical_1999}
provide tolerance against arbitrary faults (including attacks) under
some circumstances, but are limited to small networks due to poor scalability.
Proof-of-work \cite{dwork_pricing_1993,nakamoto_bitcoin:_2008}
and proof-of-stake \cite{king_ppcoin:_2012}
provide better scalability, but wasteful of computational and energy
resources.
Federated Byzantine Agreement (FBA) \cite{mazieres_stellar_2015}
is scalable and optimal, guaranteeing that the protocol only fails when
success is impossible, although it does not provide a way to evaluate
the probability of failure.
All of these protocols rely on the assumption that their cryptography
cannot be compromised.
While cryptography can be extremely resistant to technological attacks,
it can still be compromised through coercion (e.g., legal action).
In additon, the fault-tolerance of FBA depends on redundancy in the network structure
but leaves an unanswerd question: how should a network be structured to achieve
fault tolerance?
This paper addresses both of these issues by analyzing the role of network
structure on adversarial faults, without relying on assumptions of cryptographic
integrity.

{\em Multipath} routing uses multiple paths when routing a message through a
network, in contrsast to traditional {\em unipath} routing, which uses
a single path.
Multipath routing can have many benefits, including reduced congestion,
increased throughput, and more reliability
\cite{qadir_exploiting_2015}.
Many of these routing protocols offer better security 
\cite{zin_survey_2015}.
Some approaches utilize redundant paths as backups for increased
fault tolerance
\cite{alrajeh_secure_2013},
and some specificially protect against adversarial faults
\cite{kohno_improvement_2012, khalil_unmask:_2010, lou_h-spread:_2006}.
The method of Liu et al.
\cite{liu_secure_2012}
routes multiple messages first to random peers and then
to their final destination.
The butterfly algorithm we present takes a conceptually similar approach.
Most work on multipath routing has been motivated by applications related to
wireless sensor networks (WSNs),
and have thus focused on ad-hoc, unstructured networks, often having a central
base station.
The trust model presented in this paper provides a more general way to evaluate
the effect of network structure on adversarial fault-tolerance.

\section{Attack-Tolerant Networks Infrastructure}

In this section, we describe the functional properties required of an
attack-tolerant network infrastructure and how those properties translate
into constraints on network structure.
We pay special attention to a property we call
{\em stabilizing asymmetry}.

\subsection{Functional Properties}

\begin{description}
\item[Scalability]
For large scale networks it is important that the infrastructure allows for the network to grow while remaining functional. In practice, people, devises, and connections, have limited capabilities and these limitations need to be considered as part of the design of the infrastructure. 

\item [Decentralization]
Systems having single points of failure are less tolerant against faults at those
points, which both raises the likelihood of a failure and creates targets
where attackers might be able to focus their resources efficiently.
Attack-tolerant infrastructure must be decentralized in order to minimize single points of failure.

\item[Stabilizing asymmetry]
In the context of international conflict,
\cite{mack_why_1975}
observed that power imbalance usually determines the outcome of a conflict
(with the more powerful side winning)
except in the special case of {\em asymmetric conflicts}.
In asymmetric conflicts, the same level of resource expenditure yields different
results for different parties.
There are two possible cases:
the attacker's resources are either more or less effective than the defender's.
We call the latter case stabilizing asymmetry,
because it lowers the incentive to attack.
With this in mind, an attack-resistant infrastructure will benefit from a high
level of stabilizing asymmetry.

\end{description}
\subsection{Structural Properties}

We now describe the structural properties of a network that allow it to achieve the functional properties described above. 
\begin{description}
\item [Sparsity and low diameter]
To achieve scalability, networks must be {\em sparse} and have a
{\em low diameter}.
In practical settings, humans and devices have an upper limit on the number
of connections they can maintain.
In sparse networks, the number of links grows slowly as the network grows in
size, allowing the network to scale without exceeding the nodes' capacity for
links.
Similarly, low-diameter guarantees that as a network grows, a short path will
still exist between any pair of nodes.
While low diameter guarantees a path exists,
paths are only useful if an efficient {\em routing} algorithm exists
to find them.

\item [Vertex transitivity]
In vertex transitive networks, for any pair of nodes, an edge-preserving map
always exists from one node to the other.
In other words, all nodes occupy structurally indistinguishable positions
in the network. Networks with {\em vertex transitivity} are decentralized in the sense that they structure does no allow for  single points of failure.

\item [Redundancy]
Redundancy in a network refers to the extent to which nodes are connected to each other by multiple independent paths. Redundancy can help reduce single points of failure as well as
increase fault-tolerance.
One measure of redundancy is given by the ratio of edges to nodes
\cite{baran_distributed_1964}.
A single point of failure occurs when a node holds a uniquely
central position,
and can be eliminated by adding alternative
paths around the node,
which increases the network's redundancy.
Similarly, as we will later show, redundancy in a network can enable fault-tolerance techniques
that reduce the effectiveness of attacks and creating stabilizing asymmetry.
When secrecy is desired, redundancy can help by allowing messages to be
divided into parts and sent along different paths so even if one path is
compromised, the whole message is not revealed \cite{}. 

\end{description}

\subsection{Trust}

The sparsity requirement has an important consequence for network infrastructure:
as a network grows, the number of nodes will exceed the number of links a given
node can support and many pairs of nodes will only be indirectly connected,
with all paths between them passing through at least one intermediary node.
These intermediary nodes are targets for potential attacks.
More specifically, the threat model is as follows: Alice sends a message to
Bob via one or more intermediaries, but some intermediaries may try to
either block or alter the message.
The ability to tolerate these types of adversarial attacks depends on
Alice and Bob's ability to assume that some nodes have not been
compromised---to {\em trust} them.

Both centralized and decentralized approaches are commonly used to create
trust infrastructures.
Centralized approaches such as {\em public key infrastructure} (PKI)
suffer from a number of vulnerabilities
\cite{ellison_ten_2000},
which stem largely from the single points of failure inherent to
centralization.
Alternatively, the decentralized {\em web of trust} model
\cite{zimmermann_official_1995,ferguson_practical_2003}
allows individuals
to choose who they trust initially.
Trust is then extended to new individuals if they are vouched for by a
currently-trusted individual.
However, the web of trust approach assumes that trust can be extended
transitively,
which is not generally true
\cite{christianson_why_1997}.
We note that while such approaches are decentralized in the sense that no central authority holds the keys,
they do not necessarily incorporate redundancy and the network they form may not be vertex transitive. Thus they still may suffer from single
points of failure.

\section{Partial Trust}

We now propose a decentralized trust model based on weaker transitivity
assumptions than existing web of trust models.
In following sections, we will use this model along with network redundancy to minimize vulnerabilities due to single points of failure.
Our model is based on the assumption of partially-transitive trust.
While this assumption is still idealized, it is a more realistic improvement
on fully-transitive models.

Our model (Figure \ref{fig:trust-source})
is represented as an undirected graph,
with nodes representing individual
agents and edges representing mutual trust relationships.
We define a {\em trust radius} $h$ and assume that trust is transitive
up to $h$ hops away.
In other words, the nodes $v$ and $w$ trust each other if the distance between $v$ and $w$, $d(u,v)$ is less than $h$.
For a given node $v$, we call the set of nodes that $v$ trusts the {\em trusted neighborhood} $T_h(v)$,
and all nodes at exactly distance $h$ from $v$ the {\em trust boundary} $B_h(v)$:
\beq
T_h(v) &=& \left\{ w \mid d(v,w) < h \right\} \\
B_h(v) &=& \left\{ w \mid d(v,w) = h \right\}.
\eeq
We assume that $v$ is very familiar with nodes in its trusted neighbor and $v$ would know if any of these nodes were compromised by an attacher. Hence, an adversary of $v$ has no incentive to compromise a node in  $T_h(v)$. We thus assume that no nodes within the trusted neighborhood are compromised
by an adversary of $v$. However, by compromising all nodes in the trust boundary, an adversary can
ensure that all communications leaving the trusted neighborhood are compromised. 

\begin{figure}
\centerline{\includegraphics[width=2.12in,height=1.46in]{fig-alice_trusted_neigh2}}
\caption{
Illustration of the {\em partial trust} model.
Edges represent direct, mutual trust relationships.
Alice ($v$) trusts all nodes less than
$h$ hops away---her {\em trusted neighborhood} $T_h(v)$.
Beyond that, the nodes at distance $h$ form her {\em trust boundary} $B_h(v)$.
We assume adversaries are unable to compromise nodes in the trusted neighborhood.
By compromising all nodes in the trust boundary, an adversary can ensure that
all communications leaving Alice's trusted neighborhood are compromised.
}
\label{fig:trust-source}
\end{figure}

Now we consider a source $v$ (Alice) and a destination $w$ (Bob),
shown in Figure \ref{fig:trust-source-destionation}. We also consider an adversary who aims at disrupting all possible communication paths between Alice and Bob. Since the adversary cannot compromise any nodes in the trusted neighborhoods of Alicia or Bob, the only option is to compromise nodes outside of the trust neighborhoods. An adversary could, for example, compromise all possible paths between Alice and Bob by compromising either Alice or Bob's entire trust boundary. This would require compromising $\min(|T_h(v)|, |T_h(w)|)$ nodes. In a vertex transitive network, these two trust boundaries will be the same size. Alternatively, the adversary could try to compromise a small set of nodes such that all paths between Alice and Bob pass through one of these nodes. However, If disjoint paths exist connecting each node in Alice's trust boundary to
a node in Bob's, then the cut size between the two can never be smaller than
$|T_h(v)| = |T_h(v)|$, so an adversary who compromises fewer nodes cannot compromise all possible paths between Alice and Bob.

\begin{figure}
\centerline{\includegraphics[width=3.33in,height=1.46in]{fig-bob_Alice_trusted_neigh2}}
\caption{
Partial trust model with sender (Alice) and receiver (Bob).
If disjoint paths exist connecting each node in Alice's trust boundary to
a node in Bob's,
than an adversary needs to compromise at least $|T_h(v)|$ nodes to
compromise all possible paths between Alice and Bob.
}
\label{fig:trust-source-destionation}
\end{figure}

\section{Multipath Fault Tolerance}

When the assumptions of the partial trust model hold, a network can achieve
robust fault tolerance by utilizing the redundancy of multiple paths
between the sender and receiver.
Furthermore, this fault tolerance can remain robust even when faults are
adversarially chosen by an attacker who has access to more resources
than the sender and receiver.
On a structured network, multipath fault tolerance reveals not just when
an error has occured, but also provides information about which paths provide
conflicting information,
which could be used to help locate compromised nodes.

In the terminology of fault tolerance, a {\em fault} occurs when one component
of a system behaves incorrectly (e.g., a routing node blocking or
altering a message).
The result of that fault (e.g., a recipient receiving conflicting messages)
is an {\em error} state.
If the error is undetected or incorrectly corrected, the system is
said to have experienced a {\em failure} (e.g., an altered message is
accepted as authentic).
We are concerned in particular with {\em adversarial faults},
which (as opposed to random faults)
are chosen strategically (within the assumptions of the trust model)
to maximize the liklihood of a failure.
In this section, we focus on the simplest possible question:
when using multipath fault tolerance,
what is the probability that adversarial faults will cauase an undetectable
error, and thus guarantee a failure?

They key idea behind multipath fault tolerance is that several copies of
a message are sent, along multiple routes,
rather than using traditional unicast routing.
If any of the message copies differ from one another or do not arrive,
the receiver can conclude that an error has occurred,
and possibly correct it, depending on the magnitude of the error
and the desired fault tolerance properties.

In the simplest such scheme, a sender (Alice) broadcasts copies of a message
along all possible routes to the receiver (Bob).
Such an approach takes advantage of all possible redundancy in the network,
and requires an adversary (Mal) to compromise at least one node on each path
in order to create an undetectable error.
But how many nodes is that?
The answer depends on the smallest {\em vertex cut} of the untrusted
portion of the network---the smallest set of nodes than can be removed
to separate the network into two components.
Any path between Alice and Bob must pass through all such sets,
so to guarantee an undetectable error, Mal must compromise at least as many
nodes as are in the smallest vertex cut.
Alice and Bob's trust boundaries are both vertex cuts,
so the smaller of the two is an upper bound.
It is not difficult to construct networks in which the trust boundary is exactly
the smallest vertex cut,
and we will show this to be true for the networks we consider below.
We denote the size of the smallest vertex cut---the
maximum number of disjoint paths between Alice and Bob's trusted
neighborhoods---as $B$.

Realistically, the sender may only have the resources to send a limited number of
copies, or the network may not be able to handle the congestion caused by
sending all messages along all paths.
However, we will now show that a network can provide a high level of
fault tolerance with only a few copies of a message,
provided the number of disjoint paths $B$ is large.

In a more sophisticated multipath fault tolerance scheme,
Alice randomly chooses $k$ paths and sends a copy of
her message on each.
We further assume that Alice posesses a routing algorithm
(such as described in the next section)
that enables her to construct paths that
do not intersect between trusted neighborhoods.
So the maximum number of paths Alice could choose is $B$,
otherwise two paths would have to intersect in the minimum vertex cut.
Let us assume that Mal is capable of compromising up to $l$ nodes.
Mal's best adversarial strategy is now to compromise as many nodes in the smallest
vertex cut as possible,
succeeding when each of Alice's paths include one of the compromised nodes.
Since Alice chooses paths randomly,
all nodes in the vertex cut are equally likely to contain a path,
so Mal can do no better than also choosing randomly.
If $k > l$, at least one message will get through uncompromised and all
errors are detectable.
Otherwise, the probability of Mal producing an undetectable error is
the probability that all of Alice's paths contain compromised nodes:
\beq
\label{eq:pf}
p_f &=& \frac{l!(B-k)!}{B!(l-k)!}.
\eeq
Letting $k=\alpha B$ and $l=\beta B$, then applying Stirling's
approximation gives:
\begin{eqnarray}
\label{eq:pf_approx}
p_f &\approx&
\frac{\sqrt{\beta(1-\alpha)}}{\sqrt{\beta-\alpha}}
\left[
    \left( \frac{\beta-\alpha}{1-\alpha} \right)^{\alpha}
    \left( \frac{\beta}{\beta-\alpha} \right)^{\beta}
    (1-\alpha)
\right]^B.
\end{eqnarray}

Figure \ref{fig:pfail} shows the value of $p_f$
as a function of $k$ and $l$ for various values of $B$.
Equation (\ref{eq:pf_approx}) shows that while $p_f$
depends on the fractions of
redundant paths actually utilized $\alpha$ and compromised $\beta$,
it decreases exponentially in the total number of {\em possible}
redundant paths.
This result is significant because $B$ depends only on the network structure,
not on the amount of resources available to the sender or the adversary.
In other words, senders can tolerate attacks from significantly more powerful
adversaries, as long as the network structure provides large $B$.
Formally we have shown that under the partial trust assumption,
multipath fault tolerance can provide {\em asymptotic security}
against adversarial faults if $B$ can be made arbitrarily large.
Similarly, because the cost of an attack is prohibitively large
compared to the cost of protecting against an attack,
this architecture creates what a stabilizing asymmetry.
By putting attackers at a disadvantage, attacks are not only tolerated
but also disincentivized.

\begin{figure}
\centerline{\includegraphics{fig-perror}}
\caption{
The probability of an undetectable error as a function of the number of
message copies and the number of adversarial faults,
for various numbers of redundant paths.
}
\label{fig:pfail}
\end{figure}

\section{Multipath Routing on the Butterfly Topology}

\subsection{Butterfly Network Topology}

We now present a decentralized, fault-tolerant, multipath routing algorithm
on a specific topology known as the butterfly network topology
\cite{kshemkalyani_distributed_2008}.
Several variations on the butterfly network exist.
Specifically, we utilize the wrap-around butterfly.
We denote the $m$-dimensional, directed wrap-around butterfly as $\wbf(m)$:
\beq
\wbf(m) &=& (V, E_\downarrow \cup E_\rightarrow) \\
V &=& \mathbb{Z}_{m} \times \mathbb{Z}_2^m \\
E_\downarrow
&=&
\{((l,z),(l+1 (\text{mod } m),z) \} \\
E_\rightarrow
&=&
\{(l,z),(l+1 (\text{mod } m),
(z_0, \ldots, z_{l-1},z_l \oplus 1, z_{l+1}, \ldots, z_{m-1}) \},
\eeq
where $\mathbb{Z}_m$ is the set of integers modulo $m$,
and $\oplus$ represents the addition modulo 2.
Each node is associated with a level $l$ and an $m$-bit string $z$
known as {\em the place-within-level}.
There are two types of edges, shown in Figure \ref{fig:butterfly}.
Down edges ($E_\downarrow$) connect nodes sharing the same $z$ value
in a cycle of increasing level $l$.
Down-right edges ($E_\rightarrow$) also link to a node of level $l + 1$,
but one having the place-within-level equal to $z$ with the $l$th bit inverted.

The wrap-around butterfly network has a number of properties desirable for
a scalable, decentralized communication network.
\begin{description}
\item[Vertex-transitive]
Because the wraparound butterfly is vertex transitive,
it is maximally decentralized.
We will also use this property throught the proof, noting that
the problem of finding a route between arbitrary nodes $\tilde{v}$ and $\tilde{w}$
can be reduced to finding a route from node $(0,0)$ to some node $w$.
\item[Small-diameter]
For any two nodes, the length of the shortest path between them is
$O(\log N)$, where N is the number of nodes in the network.
\item[Sparse]
With a constant degree of 4, the wraparound butterfly is extremely sparse,
and can scale indefnitely without node degree becoming a limitation.
\item [Redundant] 
The number of independent paths between the $h$-trusted neighborhoods of two
nodes in the wraparound butterfly increases exponentially in $h$.
We prove this assertion below by giving an algorithm to explicitly construct
these paths.
\end{description}

\begin{figure}
\begin{center}
\includegraphics{fig-butterfly.pdf}
\end{center}
\caption{
Schematic illustration of the two types of edges in a directed butterfly network.
The node $(l,z)$ is shown as the bit string $z$ with a square around the $l$th bit.
"Down" edges increment $l$, leaving $z$ unchanged, while "down-right" edges
increment $l$ and invert the $l$th bit of $z$.
\label{fig:butterfly}
}
\end{figure}

\subsection{Routing Algorithm: Overview}

This section gives an informal overview of the multipath routing algorithm.
In a wraparound butterfly network with trusted radius $h$,
the algorithm constructs $2^h$ disjoint paths between the trusted neighborhoods
of any two nodes $v$ and $w = (l_w, z_w)$.
Because of vertex transitivity,
we can make the simplifying assumption that $v = (0,0)$ without loss of generality.
Each path is parameterized by an $h$-bit string $s \in \mathbb{Z}_2^h$ (see Table \ref{tab:routing}).
Each path cycles around the levels of the butterfly network at most twice.

During the first cycle, the place-within-level $z$ of all nodes is set such that
no two paths overlap outside the trusted neighborhoods.
During the second cycle, the place-within-level is set to its destination value
and the algorithm terminates when the destination level is reached.

More specifically, consider any node, $(l,z)$, in the path corresponding to $s$. The algorithm can be understood by dividing the $m$ bits of
$z$ into four segments: $A, B, C, D$ (see figure \ref{fig:route-overview}).
$A$ is the first $h$ bits,
and $C$ is the $h$ bits preceeding index $l_w$.
$B$ is formed by the remaining bits between $A$ and $C$,
while $D$ refers to the remaining bits after $C$.
Hence, $A, B, C, $ and $D$ occupy $h$, $l_w - 2h$, $h$, and $m - l_w$ bits of $z$, respectively
(we leave the caase of overlapping $A$ and $C$ for the formal proof).
Note that due to the construction of the network,
as we walk through any path $\{p_1, p_2, \dots\}$,
the level $l_{t+1}$ of node $p_{t+1}$ is always one higher (mod $m$)
than level $l_t$ the previous node $p_t$.
Also, $z_{t+1}$ is always the same as the $z_t$,
except for possibly the $(l+1)^{th}$ bit of $z_{t+1}$,
which can be either 0 or 1.
Hence, as the routing algorithm progresses through the levels $l$,
the segments of $z$ get updated in order. 

There are 7 stages of the algorithm to construct a path from $v$ to $w$.
Stages (1) through (4) consist of the first cycle thought the levels of
the network and stages (5) though (7) consists of the second
(potentially partial) cycle.
We start at the source node $v = (0,0)$ and begin constructing the path.
During stage (1), $A$ is set to match $s$.
During stage (2) the bits of $B$ are inverted.
During stage (3) a cyclic permutation of $s$ is placed into $C$
(the details of the permutation are necessary for the overlapping case,
but not relevant for this overview).
After this, in stages (4--7), $D$, $A$, $B$, and $C$ are set to their
destination values (see figure \ref{fig:route-overview}).
Figure \ref{fig:routing} shows an explicit example of the routing algorithm
for one value of $s$.

We now argue that in the above routing algorithm,
given $s, s' \in \mathbb{Z}_2^h$ such that $s \neq s'$,
the corresponding paths, $P_s$ and $P_{s'}$,
are disjoint when they are both outside of the trusted neighborhoods of
$v$ and $w$.
We informally refer to different segments of nodes in $s$ and $s'$ as
$A,B,C,D$ and $A',B',C',D'$, respectively.

First note that any overlap between the two paths must occur either while they
are at the same stage of the routing, or while one is at stage (1) and the other one at (5), one is at stage (2) and the other one at (6), or one is at stage (3) and the other one at (7).
This is because in any other case,
the nodes in the paths will be at different levels, $l$,
and thus they cannot overlap.
Furthermore, when a path is in stage (1) or (7),
it is in the trusted neighborhoods of $v$ or $w$,
hence we are not concerned with overlaps within stages (1) or (7). 

When both paths are in stages (2), (3), or (4), they do not overlap because
segment $A$ is set to $s$ and segment $A'$ is set to $s'$.
When both paths are in stages (5) or (6),
they do not overlap because segment $C$ is set to a cyclic permutation of $s$
and segment $C'$ is the same cyclic permutation of $s'$.
When one path is in stage (1) and the other one is in stage (5),
say $P_s$ is in (1) and $P_{s'}$ is in (5),
they do not overlap because all bits in $B$ are set to the original bits in the
source node and all bits in $B'$ are set to the inverted bits of the source node.
When one path is in stage (2) and the other one is in stage (6), they do not
overlap because at least one pair of bits in $B$ and $B'$ are still inverted
while stage (6) is being processed. 

\begin{figure}
\begin{center}
\begin{tikzpicture}[
node distance=0pt,
 start chain = A going right,
    X/.style = {rectangle, draw,% styles of nodes in string (chain)
                minimum width=10ex, minimum height=3ex,
                outer sep=0pt, on chain},
                        ]
\foreach \i in {0\ldots,{\ldots}0\ldots,{\ldots}0\ldots,{\ldots}0}% <-- content of nodes
    \node[X] {\i};
\draw[<->] ([yshift=1.5mm] A-1.north east) -- node[above=0.25mm] {$h$} ([yshift=1.5mm] A-1.north west);
\draw[<->] ([yshift=1.5mm] A-2.north east) -- node[above=0.25mm] {$l_w - 2h$} ([yshift=1.5mm] A-2.north west);
\draw[<->] ([yshift=1.5mm] A-3.north east) -- node[above=0.25mm] {$h$} ([yshift=1.5mm] A-3.north west);
\draw[<->] ([yshift=1.5mm] A-4.north east) -- node[above=0.25mm] {$m - l_w$} ([yshift=1.5mm] A-4.north west);
\draw ( A-1.west) -- node[left=5ex,minimum width=10ex] {start} ( A-1.west);
\node (B1) [inner sep=1pt,above=of A-1.north,above=5ex] {\underline{A}};
\node (B2) [inner sep=1pt,above=of A-2.north,above=5ex] {\underline{B}};
\node (B3) [inner sep=1pt,above=of A-3.north,above=5ex] {\underline{C}};
\node (B4) [inner sep=1pt,above=of A-4.north,above=5ex] {\underline{D}};
\end{tikzpicture}
\\
\begin{tikzpicture}[
node distance=0pt,
 start chain = A going right,
    X/.style = {rectangle, draw,% styles of nodes in string (chain)
                minimum width=10ex, minimum height=3ex,
                outer sep=0pt, on chain},
    Y/.style = {rectangle, draw,% styles of nodes in string (chain)
                minimum width=10ex, minimum height=3ex,
                outer sep=0pt, on chain, thick},
                        ]
\node[Y] {$s$};
\foreach \i in {{\ldots}0\ldots,{\ldots}0\ldots,{\ldots}0}% <-- content of nodes
    \node[X] {\i};
\draw ( A-1.west) -- node[left=5ex,minimum width=10ex] {1.} ( A-1.west);
\end{tikzpicture}
\\
\begin{tikzpicture}[
node distance=0pt,
 start chain = A going right,
    X/.style = {rectangle, draw,% styles of nodes in string (chain)
                minimum width=10ex, minimum height=3ex,
                outer sep=0pt, on chain},
    Y/.style = {rectangle, draw,% styles of nodes in string (chain)
                minimum width=10ex, minimum height=3ex,
                outer sep=0pt, on chain, thick},
                        ]
\foreach \i in {$s$}% <-- content of nodes
    \node[X] {\i};
\node[Y] {{\ldots}1\ldots};
\foreach \i in {{\ldots}0\ldots,{\ldots}0}% <-- content of nodes
    \node[X] {\i};
\draw ( A-1.west) -- node[left=5ex,minimum width=10ex] {2.} ( A-1.west);
\end{tikzpicture}
\\
\begin{tikzpicture}[
node distance=0pt,
 start chain = A going right,
    X/.style = {rectangle, draw,% styles of nodes in string (chain)
                minimum width=10ex, minimum height=3ex,
                outer sep=0pt, on chain},
    Y/.style = {rectangle, draw,% styles of nodes in string (chain)
                minimum width=10ex, minimum height=3ex,
                outer sep=0pt, on chain, thick},
                        ]
\foreach \i in {$s$,{\ldots}1\ldots}% <-- content of nodes
    \node[X] {\i};
\node[Y] {$\tilde{s}$};
\foreach \i in {{\ldots}0}% <-- content of nodes
    \node[X] {\i};
\draw ( A-1.west) -- node[left=5ex,minimum width=10ex] {3.} ( A-1.west);
\end{tikzpicture}
\\
\begin{tikzpicture}[
node distance=0pt,
 start chain = A going right,
    X/.style = {anchor=base, rectangle, draw,% styles of nodes in string (chain)
                minimum width=10ex, minimum height=3ex,
                outer sep=0pt, on chain},
    Y/.style = {rectangle, draw,% styles of nodes in string (chain)
                minimum width=10ex, minimum height=3ex,
                outer sep=0pt, on chain, thick},
                        ]
\foreach \i in {$s$,{\ldots}1\ldots,$\tilde{s}$}% <-- content of nodes
    \node[X] {\i};
\node[Y] {$z_{w,D}$};
\draw ( A-1.west) -- node[left=5ex,minimum width=10ex] {4.} ( A-1.west);
\end{tikzpicture}
\\
\begin{tikzpicture}[
node distance=0pt,
 start chain = A going right,
    X/.style = {anchor=base, rectangle, draw,% styles of nodes in string (chain)
                minimum width=10ex, minimum height=3ex,
                outer sep=0pt, on chain},
    Y/.style = {rectangle, draw,% styles of nodes in string (chain)
                minimum width=10ex, minimum height=3ex,
                outer sep=0pt, on chain, thick},
                        ]
\node[Y] {$z_{w,A}$};
\foreach \i in {{\ldots}1\ldots,$\tilde{s}$,$z_{w,D}$}% <-- content of nodes
    \node[X] {\i};
\draw ( A-1.west) -- node[left=5ex,minimum width=10ex] {5.} ( A-1.west);
\end{tikzpicture}
\\
\begin{tikzpicture}[
node distance=0pt,
 start chain = A going right,
    X/.style = {anchor=base, rectangle, draw,% styles of nodes in string (chain)
                minimum width=10ex, minimum height=3ex,
                outer sep=0pt, on chain},
    Y/.style = {rectangle, draw,% styles of nodes in string (chain)
                minimum width=10ex, minimum height=3ex,
                outer sep=0pt, on chain, thick},
                        ]
\foreach \i in {$z_{w,A}$}% <-- content of nodes
    \node[X] {\i};
\node[Y] {$z_{w,B}$};
\foreach \i in {$\tilde{s}$,$z_{w,D}$}% <-- content of nodes
    \node[X] {\i};
\draw ( A-1.west) -- node[left=5ex,minimum width=10ex] {6.} ( A-1.west);
\end{tikzpicture}
\\
\begin{tikzpicture}[
node distance=0pt,
 start chain = A going right,
    X/.style = {anchor=base, rectangle, draw,% styles of nodes in string (chain)
                minimum width=10ex, minimum height=3ex,
                outer sep=0pt, on chain},
    Y/.style = {rectangle, draw,% styles of nodes in string (chain)
                minimum width=10ex, minimum height=3ex,
                outer sep=0pt, on chain, thick},
                        ]
\foreach \i in {$z_{w,A}$,$z_{w,B}$}% <-- content of nodes
    \node[X] {\i};
\node[Y] {$z_{w,C}$};
\foreach \i in {$z_{w,D}$}% <-- content of nodes
    \node[X] {\i};
\draw ( A-1.west) -- node[left=5ex,minimum width=10ex] {7.} ( A-1.west);
\end{tikzpicture}
\end{center}
\caption{
Progression of place-within-level $z$ as the multipath routing algorithm
cycles through the levels of the butterfly network.
}
\label{fig:route-overview}
\end{figure}

\begin{table}%
\caption{Butterfly Multipath Routing Variables\label{tab:routing}}
{
\begin{tabular}{|l|l|}
\hline
NAME & VARIABLE \\\hline
butterfly dimension & $m \in \mathbb{Z}_+$ \\\hline
node level & $l \in \mathbb{Z} : 0 \leq l < m$ \\\hline
node place within level & $z \in \mathbb{Z}_2^m$ \\\hline
trust radius & $h \in \mathbb{Z} : 1 \leq h \leq \lfloor m/2 \rfloor$ \\\hline
path index & $s \in \mathbb{Z}_2^h$ \\\hline
\end{tabular}
}
\end{table}%

\begin{figure}
\begin{center}
\includegraphics{fig-routing.pdf}
\end{center}
\caption{
An example of one path as constructed by the proposed multipath
routing algorithm.
The path is shown for $s = 10_2$
and $w = (6, 0110111_2)$.
\label{fig:routing}
}
\end{figure}

\subsection{Routing Algorithm: Specification and Proof}

We now specify the multipath routing scheme and prove
that provides $2^h$ disjoint paths between
the $h$-hop trusted neighborhoods of any two nodes in an $m$-bit
wrap-around butterfly network.
Utilizing vertex transitivity, we label the source node as
$(l^{(0)}, z^{(0)}) = (0, 0)$ and denote the destination node as $w = (l_w, z_w)$,
without loss of generality.

Let $s$ be an $h$-bit binary string with $s_i$ denoting the bit at index $i$.
There are $2^h$ such strings.
Let $v_s^{(t)} = (l^{(t)}, z^{(t)})$ be the node at position $t$
in the path parameterized by $s$.
For convenience,
we will omit the subscript $s$ when it is obvious from context.
We define three distinct partitions of $m$-bit binary strings.
Let $Q_{v^{(0)}}$ ($\overline{Q_{v^{(0)}}}$) be the set of $m$-bit
strings in which the bits at
all indeces $h \leq i < l_w - h$ match (do not all match) those of $z^{(0)}$.
Note that $Q_{v^{(0)}}$ is trivially all $m$-bit strings if $l_w < 2h$.
Let $R_s$ ($\overline{R_s}$) be the set of $m$-bit strings with the lowest $h$
bits all matching (not all matching) the bits of $s$.
Let $S_s$ ($\overline{S_s}$) be the set of $m$-bit strings with the $h$ bits
preceeding index $l_w$ all matching (not all matching) the bits of $\tilde{s}$,
where $\tilde{s}$ is a cyclic permutation of $s$ defined below.
\beq
Q_{v^{(0)}} &=&
\{z \in \mathbb{Z}_2^m \mid \;
\neg \exists i \in \mathbb{Z} : h \leq (i < l_w - h) \land z_i \neq z_i^{(0)} \}
\\
\overline{Q_{v^{(0)}}} &=& \mathbb{Z}_2^m \setminus Q_{v^{(0)}}
\\
R_s &=&
\{z \in \mathbb{Z}_{2}^m \mid \;
\forall i \in \mathbb{Z}_2^h z_i = s_i \}
\\
\overline{R_s} &=& \mathbb{Z}_2^m \setminus R_s
\\
S_s &=&
\{z \in \mathbb{Z}_{2}^m \mid \;
\forall i \in \mathbb{Z}_2^h z_{(l_w - h + i)} = \tilde{s}_i \}
\\
\overline{S_s} &=& \mathbb{Z}_2^m \setminus S_s
\\
\tilde{s}_i &=& s_{(i + l_w) \text{ mod } h}.
\eeq
We will make use of the fact that:
\beq
s \neq s^\prime &\implies&
S_s \cap S_{s^\prime} = R_s \cap R_{s^\prime} = \emptyset.
\eeq

Routes are contructed in 7 stages.
The network topology dictates that $l^{(t+1)} = l^{(t)} + 1$ (mod $m$),
so we let $l = t$ (mod $m$).
and that $z^{(t+1)}$ is equal to $z^{(t)}$ with or without the bit in index
$l^{(t)}$ inverted, depending on whether the down or down-right edge was
taken at step $t$.
\begin{description}
\item[Stage 1: ($0 \leq t < h$)]{
Down or down-right edges
are chosen such that the $t$th bit of $z^{(t+1)}$ is equal to the $t$th bit
of $s$.
Throughout Stage 1, all nodes are within the sender's trusted neighborhood.
Throughout Stage 1, $z^{(t)} \in Q_{v^{(0)}}$.
At the end of Stage 1, $z^{(h)} \in S_s$, and $z^{(t)}$ will remain so until the level cycles to $0$ at $t = m$.
}
\item[Stage 2: ($h \leq t < l_w - h$)]{
Edges are chosen to make the $t$th bit of
$z^{(t+1)}$ the inverse of the $t$th bit of $z^{(0)}$.
Note that this stage does not occur when $l_w < 2h$.
If this stage occurs, then $z^{(t)} \in \overline{Q_{v^{(0)}}}$ until these
levels are reached again in stage 6.
}
\item[Stage 3: ($l_w - h \leq t < l_w$)]{
The bits of $z^{(t)}$ are chosen to match $\tilde{s}$,
such that after the stage is complete, $z^{(t)} \in R_s$.
}
\item[Stage 4: ($l_w \leq t < m$)]{
Paths are chosen such that the $t$th bit of $z^{(t+1)}$ matches that of the
destination node $z_w$.
This stage will not occur if $l_w > m - h$.
}
\item[Stage 5: ($m \leq t < m + h$)]{
There are two cases.
If $2h < l_w < m - h$,
then there is no overlap between the indeces defining $R_s$ and $S_s$.
In this case, the first $h$ bits of $z^{(t)}$ are set to
match $z_w$.
Otherwise there is some overlap between the indeces defining $R_s$ and
$S_s$.
In this case, the each of the first $h$ bits of $z^{(t)}$ is either kept the
same if $l_w - h \leq l < l_w$, or set to the corresponding bit of $z_w$
otherwise.
In this stage and after, $z^{(t)}$ is no longer guaranteed to be in $R_s$.
However, $z^{(t)}$ remains in $S_s$ during and after this stage.
}
\item[Stage 6: ($m + h \leq t < m + l_w - h$)]{
In this stage, edges are chosen to set the bits of $z^{(t)}$ to their
corresponding value in $z_w$.
$z^{(t)} \in \overline{Q_{v^{(0)}}}$ throughout this stage,
but not afterwards.
}
\item[Stage 7: ($m + l_w - h \leq t < m + l_w$)] {
The $h$ bits of $z^{(t)}$
preceeding index $l_w$ are set to match $z_w$.
All nodes in this stage are within $h$ hops of $w$ and thus in its trusted
neighborhood.
After this stage, $v^{(m + l_w)} = w$ and routing is complete.
}
\end{description}

For all $2^h$ of these paths, the components excluding nodes within $h$ hops
of $v$ or $w$ (in their trusted neighborhoods) are pairwise disjoint,
which we now prove.
Nodes from two paths can only coincide if their levels are the same.
Nodes which share a level must either be in the same stage, or 4 stages
apart.
Let ($a$,$a^\prime$) denote a pair of sub-paths corresponding to stage $a$ of
one path and stage $a^\prime$ of another.
Excluding paths that intersect in their trusted neighborhoods, (1,1) and (7,7),
we have reduced the list of possible intersections to the following cases:
(2,2), (3,3), (4,4), (5,5), (6,6), (1,5), (2,6), and (3,7).
Nodes in stages 2--4 belong to $R_s$ so cannot overlap with any stage 2--4
nodes from another path, eliminating (2,2), (3,3), and (4,4).
Similarly, nodes in stages 4--6 belong to a unique $S_s$,
eliminating (5,5) and (6,6).
Nodes in stage 1 belong to $Q_{v^{(0)}}$ while those in stage 5 belong in
its complement, eliminating (1,5).
Similarly, for all $l$ in stage 2, $z^{(l)}$ is equal to $z^{(0)}$,
while in stage 6, $z^{(l)}$ is the inverse, eliminating (2,6)
This leaves only (3,7), a collision which can occur only for only one path
(with $s$ matching the first $h$ bits of $z_w$), and which enters the trusted
neighborhood in stage 3.
For this single path, we can proceed directly from stage 2 to stage 7,
eliminating the last possible collision.

Thus, we have shown that assuming the partial trust model, with trust transitive
for $h$ hops, we can construct $2^h$ paths on a wraparound butterfly topology
which do not intersect outside the trusted neighborhoods of the source and
destination.
This is a lower bound on the value $B$ in Equation (\ref{eq:pf}),
showing that the decentralized, redundant, structured networks such as the
butterfly can have a very low probability of failure when faced with
adverarial faults, even from a very powerful attacker.

\section{Discussion}

While decentralized architectures have recieved much attention,
methods for classifying, and evaluating decentralization
are in their infancy.
The trust model and multipath fault tolerance technique proposed in this
paper provide a practical fault tolerance architecture for applications when
network structure can be imposed,
as well as a more general method for evaluating the role of network structure
on fault tolerance.

Fault-tolerant network infrastructures have many direct applications.
Areas such as cryptocurrency, secure multiparty computation, and wireless
sensor networks have immediate need for scalable, fault-tolerant
infrastructures.
Many Internet services---email, social networks, cloud storage---are still
highly centralized and vulnearble to technical and non-technical attacks.
Decentralized fault tolerance is one approach to securing these important
services and making networked communication safer.

While the primary purpose of our proposed fault tolerance technique is the
threat of message blocking or corruption, it can provide some improvements
to secrecy and privacy.
For example, if a message is broken into parts and each part is sent along
a different random subset of routes, an attacker who succesfully compromises
one part of the message will still be unable to compromise the rest.

Our present contributions are limited in several ways.
The partial trust model, while making fewer assumptions than web of trust
approaches, still depends on some transitivity of trust,
which may be a questionable assumption in many cases.
Better trust models are still needed.

The primary limitation of our structured network approach is that
link structure is imposed rather than self-organized.
This requirement may be appropriate for appliations such as
wireless sensor networks,
or slow-changing institutional infrastructure,
in which the architect has considerable control over link structure.
But in general, it remains an open question whether self-organized networks
can be restructured, or buit from the ground up in such a way that
structured multipath fault tolerance technqiues can be applied.
Similarly, structured network topologies such as the butterfly are defined
only for specific values.
Although these values can be as large as desired,
how to apply methods such as ours to networks with an intermediate number
of nodes remains an open question.

\section{Conclusion}

We have developed a partial trust model which enables a formal analysis
of the effect of decentralized network structure on fault tolerance properties.
We have also proposed a general technique for achieving and evaluating adversarial
fault tolerance in a structured network with partial trust.
We found that the probability of an adversary causing an undetectable error
decreases exponentially as the number of redundant paths between the
neighborhoods of the sender and receiver grows.
This property puts attackers at a disadvantage, creating a stabilizing
asymmetry which disincentivizes attack.
We also presented a specific implementation of the multipath fault tolerance
technique on a wraparound butterfly network topology and proved that the
number of redundant paths grows exponentially with number of network hops
that can be trusted.
Our results are directly applicable to architectures in which the link structure
can be imposed by the architect,
and provide a formalism that can be used more generally to evaluate the role
of network structure and redundancy on the fault tolerance of decentralized
networks.

%ACKNOWLEDGMENTS are optional
\section{Acknowledgments}
Anonymized for blind review.

%
% The following two commands are all you need in the
% initial runs of your .tex file to
% produce the bibliography for the citations in your paper.
\bibliographystyle{abbrv}
\bibliography{paper}  % sigproc.bib is the name of the Bibliography in this case
% You must have a proper ".bib" file
%  and remember to run:
% latex bibtex latex latex
% to resolve all references
%
% ACM needs 'a single self-contained file'!
%
%
\end{document}
