\documentclass[twocolumn,10pt]{article}

\usepackage[utf8]{inputenc}
\usepackage{amsmath}
\usepackage{amssymb}
\usepackage{amsthm}
\usepackage{graphicx}
\usepackage{times}
\usepackage{geometry}
\usepackage[ruled,vlined]{algorithm2e}

\usepackage[switch]{lineno}
\linenumbers

\newtheorem{definition}{Definition}
\newtheorem{property}{Property}
\newtheorem{claim}{Claim}

\DeclareMathOperator{\unif}{unif}
\DeclareMathOperator{\sample}{sample}
\DeclareMathOperator{\setcount}{count}
\DeclareMathOperator{\mode}{mode}
\DeclareMathOperator*{\argmax}{arg\,max}
\DeclareMathOperator*{\argmin}{arg\,min}

\usepackage{geometry}
\geometry{left=1.5cm,right=1.5cm,top=2cm,bottom=2cm}
\setlength{\columnsep}{0.5cm}

\title{Supplementary Information: \\
Small interlocking groups improve mass deliberation\\in the presence of strong social influence}
\author{
%Edward L. Platt\\
%University of Michigan\\
%elplatt@umich.edu
%and
%Herminio Bodon\\
%Northwestern University\\
%HerminioBodon2020\\@u.northwestern.edu
%\and
%Daniel M. Romero\\
%University of Michigan\\
%drom@umich.edu
Authors Redacted \\
for \\
Double-Blind Peer-Review
}
\date{\today}

\begin{document}

\maketitle

\section{Simulation Procedure}
We begin with a network $(V,E_t)$. The vertices $V$ correspond to agents. The edges $E_t$ allow agents to exchange information with their immediate neighbors at time $t$.
Agents collaborate to find a solution $s$ from a space of solutions $\mathcal{S}$ that maximizes an objective function $Q(s)$.
We use binary strings of length $d$ as our solution space: $s \in \mathbb{Z}^d$.
At any one time $t$, each agent $v$ has exactly one preferred solution $s_{v,t}$.
We generate a tunably rugged objective function $Q(s)$ using an NK-Model \cite{kauffman_towards_1987} (see Section \ref{subsec:task}) with N=15, K=6, exp=8.

\section{Network Topologies}

We use networks to represent constraints on who talks to whom. Each agent is represented by a vertex, and two agents are able to interact when their vertices are connected by an edge. In this paper we are concerned with network deliberation, which takes place on interlock networks. Interlocks are composed of small cliques (i.e., pods) with some vertices belonging to multiple cliques. For the interlock networks in this paper, each vertex belongs to exactly one pod at a time, with pod membership periodically reassigned to achieve multiple membership. As a control, we also consider networks representing conventional deliberation.


\section{Learning Strategies}
\label{sec:learning}

% Exclude bitwise majority from paper
\iffalse

\subsection{Bitwise Majority}
Instead of evaluating entire solutions on their popularity, it is also possible to evaluate the popularity of each component of a solution, i.e., each bit in the bit string \cite{platt_network_2018}.
We refer to this strategy as {\em bitwise majority} social learning.
As with previous strategies, bitwise majority can result in ties for individual bits, which result in ties for the resulting full solution.

\begin{definition}
The bitwise majority strategy $\mathcal{L}_{bitwise}$ is defined as:
\begin{eqnarray}
S_i &=& \bigcup_{x \in S} \{x_i\} \\
\mathcal{L}_{bitwise}(S)
&=& \mode(S_1) \times \mode(S_2) \ldots \times \mode(S_d),
\end{eqnarray}
where $x_i$ is the $i$th element of the binary string $x$,
$S_i$ is a multiset, and $\mode(S)$ returns a set containing the mode or modes of $S$, and $\times$ is the Cartesian product.
\end{definition}
As with conform, the bitwise majority strategy does not incorporate any new information about solution quality.
Bitwise majority also has the notable property of being able to produce novel solutions that vary significantly from any previously seen.

% End of bitwise majority
\fi

\section{NK Model}
\label{subsec:task}
\section{Supplemental Results}



\bibliography{references}
\bibliographystyle{plain}

\end{document}
