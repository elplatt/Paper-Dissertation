The past several decades have seen the emergence of a networked information economy (Benkler, 2006). This new type of economy is characterized by fast, cheap, and bidirectional global communication, thanks to the internet, resulting in an increase in knowledge work where creativity has become the limiting resource in many endeavors. While the internet and the networked information economy have rightly been identified to hold great potential for civic participation, the past several decades have seen a drastic decline in voter turnout, political knowledge, participation in service organizations, and other indicators of civic engagement in the United States (Putnam, 2000; Ackerman \& Fishkin, 2002). Civic engagement is crucial for the development of the social capital and trust that enable collective action, which in turn allows communities to advocate for their interests and build sustainable regimes to manage resources (Ostrom, 2000). Why has the potential of the internet to enable civic participation and collective action gone largely unfulfilled? What can be learned from relatively successful internet-enabled collaborations? And what questions still need to be answered to best harness the internet for collective action?

Specifically, I will examine what is know about the many steps between large-scale communication and large scale collective action. Wikipedia and other forms of commons-based peer production serve as a key and relatively successful example (Benkler, 2002). These examples suggest that the internet can enable collective action at a global scale, if the massive potential for communication and production can be combined with a means of filtering and accreditation. Fundamentally, this is a problem of (very) large scale decision-making. I review the literature traditional collective action, commons-based peer production, collective intelligence, and social choice theory to identify common themes relevant to internet-enabled decision-making for collective action.

Large-scale decision-making is difficult and emphasis typically shifts from discourse and deliberation to voting as groups grow in size (Ackerman \& Fishkin, 2002; Gonz\'alez-Bail\'on, 2010). But the ability of voting to achieve cooperation is limited by paradoxes and impossibility results from social choice theory (Brandt et al., 2012; Arrow, 1950; Condorcet, 1785). Instead of abandoning discourse and deliberation, in this prelim I examine the importance of these modes of participation in commons-based peer production and collective action. I also emphasize that the limitations of voting can potentially be overcome by the ability of discourse and deliberation to build consensus, trust, and cooperation (Habermas, 1964; Ostrom, 2000; Geiger, 2009). I suggest that techniques for large-scale deliberation could create more effective large-scale decision-making and more effective use of the internet for collective action.

In addition to my literature review, I propose an experiment to evaluate one approach to large-scale deliberation. Across different fields, the importance of small, overlapping groups emerges repeatedly as important for building trust, propagating and filtering information, and creating non-hierarchical structure (Ostrom, 2010; Benkler, 2006; Geiger 2009; Putnam, 2000; Freeman, 1972; Gray et al., 2015). However, there has not been a systematic, quantitative study of how the network structure of overlapping groups impacts their ability to enable large scale cohesion and discourse. I propose networked deliberation as a way to combine small groups into very large scale deliberations, and describe an experiment to test its effectiveness. I transfer quantitative measures from social choice theory to track the development of consensus over the course of deliberation and quantitatively compare the effectiveness of different networks. Elinor Ostrom wrote that
``a core goal of public policy should be to facilitate the development of institutions that bring out the best in humans (2010).'' In this prelim, I ask how information and communication technology can do the same.

Chapter 2 reviews theories of collective action and collective intelligence. Chapter 3 reviews case studies of large scale collective actions, both online and off. Chapter 4 describes my proposed research.
