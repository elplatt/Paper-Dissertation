\section{Anthropological Studies}
Anthropologists have identified and studied several human societies that operate cooperatively, including the Yir Yoront (Sharpe, 1958) and the Mardujarra (Tonkinson, 1988), both among the indigenous peoples of Australia. One common feature of such cooperative peoples is mutual dependence. For example, when scarce resources and unpredictable weather make cooperation necessary for survival. Mutual dependence can be formed by two-way flows of obligations (Tonkinson, 1988). Or mutual dependence can be the result of complex non-hierarchical structure, as Sharpe (1958) describes: ``A [Yir Yoront] man has no dealing with another man (or woman, either) on exactly equal terms. And where each is at the same time in relatively weak positions and in an equal number of relatively strong positions, no one can be either absolutely strong or absolutely weak. A hierarchy of a pyramidal or inverted-Y type to include all the men in the system is impossible." In the Yir Yoront, these structures are based on geographically dispersed kinship obligations. Just as in commons-based peer production, cooperation is achieved through dispersed collaboration and overlapping sub-groups (Benkler, 2002), forming a heterarchy (Crumley, 1995). These non-hierarchical structures allow conflict resolution through reverse dominance hierarchies (Boehm et al., 1993) based on norms and deontic decision-making, just as observed in Western cooperative regimes (Ostrom, 2000).

\section{Experimental Studies}
Small scale networked social learning experiments have shown that networked groups can solve very difficult problems, but the relationship between network structure, task-type, and group performance is still not fully understood. In a series of lab experiments, Kearns (2012) has found that very difficult problems, such as graph coloring and unanimous consensus, can be solved by networked groups. In graph coloring tasks, the performance of low-connectivity cycle networks was improved by adding shortcuts to increase connectivity, while the same changes made it more difficult for groups to solve a consensus task. Mason \& Watts (2012) similarly found that high-connectivity, efficient networks were preferable for experimental groups of problem solvers seeking to maximize an NK model objective function. In one notable exception, Kearns (2012) found that self-organized groups tended to perform poorly in networked bargaining experiments, and that these networks were differentiated from others by highly skewed degree distributions.

A novel form of large-scale survey, the wiki survey, has been demonstrated to be effective at eliciting both preferences and new ideas (Salganik \& Levy, 2015). Participants are presented with pairs of proposals, and prompted to either choose one or submit a third alternative. The granular, dispersed nature of the task is more amenable to peer production (Benkler, 2002) and adapts to collect as much or as little information as participants are willing to provide. Salganik and Levy (2015) partnered with New York City and the OECD to conduct trials of approximately 1,500 participants and 30,000 pairwise comparisons each. Over the course of the surveys, the number of unique proposals increased between 5 and 10 times. New proposals included new ideas, but also re-framings of existing ideas. These findings suggest a combination of social learning and discursive processes. The wiki survey is a primary inspiration for the experiment in this prelim.

\section{Wikipedia}
The persistent record of activity on Wikipedia and other wikis has made it possible for researchers to analyze the evolution of editor behavior over long timescales. On Wikipedia, several researchers have noted changes in the community behavior starting around 2004 and ending around 2007, the so-called ``golden era'' (Keegan \& Fiesler, 2017; Kittur et al., 2007, Forte \& bruckman, 2008). In an early study of Wikipedia, Vi\'egas \& Wattenberg (2004) found many of Ostrom’s (2000) design principles for self-organized cooperation. Policies such as ``neutral point of view (NPOV) are visibly recorded and actively discussed by any interested contributors. History logs allow for monitoring. The ability to revert harmful edits means no permanent damage can be done, making graduated sanctions feasible. A later study (Vi\'egas et al., 2007) found that criteria had become stricter, e.g., for ``featured'' articles, and that automated tools were performing some of the quality control and accreditation tasks originally performed by humans. However, the same study noted that editors had developed templates allowing quality control issues to be flagged and addressed at different times by different people, enabling dispersed collaboration. These developments are signs of a change in the nature of decentralized work on Wikipedia.

Based on interviews, Forte \& Bruckman (2008) describe a shift from individual decision-making to deliberative and committee-based decision-making, and conclude that Wikipedia has become less centralized. Similarly, Kittur et al. (2007) found a shift from direct work on articles to indirect work on policy and anti-vandalism. Keegan \& Fiesler (2017) examined the edit history of Wikipedia’s policy pages, finding that after 2007, edits to policy pages declined in favor of discussion on policy talk pages. Kittur et al. (2007) found that an increasing fraction of edits were made by newer users, but that content was primarily created by long-standing users. Similarly, in a study of Wikia wikis, Shaw and Hill (2014) found that it became more difficult for new users to become editors, that existing admins used their powers more, and that new users were more likely to have their edits reverted. These studies indicate ways in which wikis become both more and less centralized over time. Overall, Wikipedia has responded to an influx of users by creating a more decentralized structure based on committees, while keeping power somewhat centralized among earlier editors through less policy flexibility, a high bar for creating new content, and higher inequality between formal admin positions and non-admin editors.

Wikipedia has also been studied to gain a better understanding of the factors that influence teamwork at a large scale. Kittur \& Kraut (2008) quantified implicit and explicit coordination (gini coefficient of editors’ edit counts and number of talk page edits, respectively), finding that both types were helpful in the early stages of an article, and that larger teams relied on implicit coordination to see any benefit from their greater resources. Romero et al. (2015) found that higher status articles also relied on more implicit coordination and that crowded articles, having many editors relative to the size, exhibit more explicit coordination. Kittur et al. (2009) also studied the role of task interdependence, for example: improving coverage by adding content (low-interdependence) or improving readability by synthesizing many existing contributions (high-interdependence). They found that articles with larger number of editors showed the benefit primarily in low-interdependence tasks, again concluding that coordination is necessary for larger groups to take advantage of their additional resources. Larger group can bring not just more time, but also more diversity and experience to a project. Robert \& Romero (2015) found that more diverse and experienced teams were better able to take advantage of large group size. Platt \& Romero (2018) found that when editors of a WikiProject interact with a fewer coeditors, those projects both have more high-quality articles and improve article quality more quickly. In summary, Wikipedia appears to be able to take advantage of its large number of contributors, in part, because those editors self-organize into small, diverse sub-groups who use implicit coordination to reduce the necessity for explicit coordination.

\section{Email Lists and Forums}
Even before the web, the internet enabled online communities in the form of email lists and forums. Franco et al. (1995) studied the course of a ``flame war'' (a series of angry replies) on an email list. The event began when a private reply containing criticism was assumed to be public and the recipient responded publicly. Franco et al. found that high prestige users were more likely to receive replies, and that their messages could change the tone of the conversation. In this example, the messages transitioned from divisive to unifying, allowing the community to publicly develop and enforce its principles. Forums, whether one Usenet or the web, offer similar functionality to email lists but sometimes enable additional features. For example, the web forum Slashdot has formal moderation positions, as well functionality for about 90% of users to review moderators (Benkler, 2002). Such features are examples of how principles for effective cooperation can be codified through technology, such as Ostrom’s design patterns (2000), meta-norms (Axelrod, 1997a), and reverse dominance hierarchies (Boehm et al., 1993; Crumley 1995). Forums have also been studied as part of the public sphere. For example, by measuring comment count and comment depth, as well as using a principle component analysis, Gonz\'alez-Bail\'on (2010) found that political discussions on Slashdot were high on both argumentation and representation (Ackerman \& Fishkin, 2002), suggesting the type of rich discourse needed for strong, participatory democracy.

\section{Organizations and Social Movements}
Both organizations and social movements have started using internet-based platforms to enable large-scale cooperation. Buurtzorg Nederland, a non-profit Dutch home-care provider employs 8000 nurses divided into 700 self-managed teams (Gray et al., 2015). Burtzorg uses a custom web application to enable coordination within and between teams and achieves higher patient satisfaction for the same expenditure as other firms. In the political sphere, the German Pirate party used a delegative democracy tool called LiquidFeedback from 2011 to 2014 (King et al., 2015). LiquidFeedback allows individuals to either vote directly on proposals or delegate their vote to a proxy (who might then delegate to another person). Using the voting and delegation records, King et al. were able to evaluate the evolution of political power over time, finding that power inequality increased but that ``super-voters'' with high power tended to vote with the majority of other voters, seldom using their power to influence results. On a smaller scale, DeTar (2013) created the InterTwinkles tool for affinity consensus decision-making in small groups (e.g., housing cooperatives), finding that the tools were often used for dispersed ideation and deliberation between face-to-face meetings. DeTar also noted the importance of technical architecture that supports the creation of ad-hoc groups within a community. Finally, Gonz\'alez-Bail\'on and Wang (2016) studied the Twitter networks of the Indignados movement in Spain and the Occupy movement in the US. They found that after the face-to-face events ended, their latent networks remained active online. They also identified brokers that bridged the two movements, but noted that only about 10% of bridges were ``active'' and using hashtags relevant to both movements.

