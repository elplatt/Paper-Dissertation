We can allow for more realistic transitivity assumptions and incorporate
network structure into a trust model with one key insight:
even if no single path between a sender and receiver is fully trusted,
multiple copies of a message can be sent along different paths and compared
by the receiver to detect and correct errors.
However, in the presence of an attacker, there is only benefit in sending an
additional message along a path if that path does not share untrusted
single points of failure with any of the existing message paths
(otherwise, the adversary can compromise both messages
by causing a single fault).
Following \cite{reiter_resilient_1998},
we call paths that do not share any untrusted single points of failure
{\em independent}.
The maximum number of independent paths represents the effective redundancy that
can be utilized by any redundancy-based fault tolerance scheme.



\subsection{Functional Properties}

\begin{description}
\item[Scalability:]
For large-scale networks, it is important that the infrastructure allows
for the network to grow while remaining functional.
In practice, people, devices, and connections, have limited capabilities
and these limitations need to be considered as part of the design of the
infrastructure. 

\item [Decentralization:]
Systems having single points of failure are less tolerant against faults at
those points.
The existence of such points not only increases the likelihood that an attack
will succeed,
but also incentivizes attack by presenting effective targets.
In order to minimize single points of failure,
attack tolerant infrastructures must use {\em both} decentralized protocols and
decentralized network structures.

\item[Stabilizing asymmetry:]
In the context of international conflict,
{\em asymmetric conflicts} allow a
less powerful party to have an advantage over a more powerful party
\cite{mack_why_1975}.
In asymmetric conflicts, the same level of resource expenditure yields different
results for different parties;
the attacker's resources are either more or less effective than the defender's.
We call the latter case stabilizing asymmetry,
because it reduces an attacker's power relative to their target.
With this in mind, an attack-resistant infrastructure will benefit from a high
level of stabilizing asymmetry.

\end{description}
\subsection{Structural Properties}

We are specifically concerned with network structure-based approaches to
fault tolerance.
In networks, specific structural properties are required to achieve the
functional properties described above.
\begin{description}
\item [Sparsity and low diameter:]
To achieve scalability, networks must be {\em sparse} and have a
{\em low diameter}.
In practical settings, humans and devices have an upper limit on the number
of connections they can maintain (e.g., Dunbar's number
\cite{dunbar_neocortex_1992}).
In sparse networks, the number of links grows slowly as the network grows in
size, allowing the network to scale without exceeding the nodes' capacity for
links.
Similarly, a low diameter guarantees that as a network grows, a short path will
still exist between any pair of nodes.
While low diameter guarantees a path exists,
paths are only useful if an efficient {\em routing} algorithm exists
to find them.

\item [Uniform centrality:]
There are many ways to measure node centrality in networks
\cite{freeman_centrality_1978}.
The more uniform these measures are across nodes, the more decentralized
a network is.
Centrality is minimized in {\em vertex transitive} networks,
for which an edge-preserving map
always exists from any node to any other node.
In other words, all nodes occupy structurally indistinguishable positions
in the network.

\item [Redundancy:]
In a network, redundancy refers to the existence of multiple disjoint
paths between nodes or components.
Redundancy can help reduce single points of failure and decrease
centralization.
One measure of redundancy is given by the ratio of edges to nodes
\cite{baran_distributed_1964}.
A single point of failure occurs when a node holds a uniquely
central position.
When alternative paths are added to bypass central nodes,
the network becomes more redundant.

\end{description}
