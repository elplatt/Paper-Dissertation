As suggested in the previous sections, The previous sections have hopefully made it clear that large-scale communication is necessary for large-scale collective action, but it is not sufficient. This research proposal addresses an additional requirement: how can groups use the internet to make decisions as part of large-scale collective action? The internet’s potential is not just in large-scale collective action, but in large-scale, decentralized, non-hierarchical collective action. Traditionally, large-scale collective action has been organized through coercive dominance hierarchies, e.g., states and firms. Findings from studies of collective intelligence and social learning suggest that non-hierarchical decision-making can avoid informational bottlenecks and biases found in hierarchical organizational structures (Hayek, 1945; Benkler, 2006; Jackson \& Golub, 2012). Even in democratically-run groups relying on voting, coercion is necessary to ensure the compliance of dissenting members, and due to the Condorcet paradox (Condorcet, 1785) and Arrow’s impossibility theorem (Arrow, 1950), winners may not even represent the group’s preferences very well.

Without coercion, the only way to ensure the members of a group comply decisions is to ensure that decisions have broad support (Ostrom, 2000). Broad support is often referred to as “consensus,” which has become in imprecise term with multiple meanings, so I will prefer “concurrence” when referring to the state of general agreement. Discourse and deliberation can identify points of common agreement and even strengthen concurrence through mutual accommodation (Anderson, 2006; Habermas, 1964; Geiger, 2009). But effective deliberation has traditionally been difficult at large scales (Ackerman \& Fishkin, 2002; Gonz\'alez-Bail\'on et al., 2010). Both offline (Putnam, 2000) and online (Benkler, 2006) deliberative communities exhibit small, overlapping sub-groups, creating a “filtering and transmission backbone” (Benkler, 2006). These non-hierarchical structures (Crumley, 1995) allow for the benefits of small group dynamics while enabling ideas to quickly propagate from group to group. I hypothesize that the specific network structure of these sub-groups will influence how effectively large-scale deliberations can generate broad agreement, and propose an experiment to test this hypothesis.

\section{Related Work}
Motivated by falling levels of civic participation (Putnam, 2000) scholars have proposed several designs for facilitating deliberation, both online and off. Face-to-face examples include deliberative polling (Fishkin et al. 2000), citizens’ juries (Crosby, 1995), planning cells (Dienel \& Renn, 1995), consensus conferences (Sclove, 1995), and deliberation days (Ackerman \& Fishkin, 2002). These highly-structured approaches generally involve pre-deliberation and post-deliberation surveys combined with a combination of reading material, lectures, small group discussions, and large plenary assemblies. While potentially effective at changing opinions, Schkade et al. (2007) found that structured deliberation could also increase polarization. Such face-to-face events are also, by nature, difficult to conduct at a large scale. 

Many internet-based systems have been developed to aid decision-making, with some of them enabling various levels of deliberation. Wiki surveys (Salganik \& Levy, 2015) allow survey respondents to submit new proposals and consider proposals from other respondents. The InterTwinkles suite of tools for consensus decision-making includes a range of tools for deliberation and voting (DeTar, 2013). Loomio (Jackson \& Kuehn, 2016) combines a forum with a flexible interface for consensus decision-making.

\subsection{Deliberation and Preferences}
Deliberation plays several important roles in democratic decision-making. It is a social learning process, allowing members of a group to exchange information. Social learning allows decision-makers to learn of options they had not yet considered and helps them re-evaluate the efficacy of known options at producing their desired outcomes. Deliberation is also a discursive process, potentially changing a decision-maker’s desired outcomes through social interaction (Habermas, 1964; Anderson, 2006), and by building trust that enables cooperation (Ostrom, 2000; Axelrod, 1997a).

The ability to change individual preferences through deliberation suggests a way to avoid the complications of Arrow’s impossibility theorem, and to create concurrence within large collective-action groups. However, methods are necessary to determine if and when deliberation actually achieves this goal. I will rely on quantitative measures from social choice theory (described in Methods) to measure the how concurrence changes over the deliberative process.

RQ1. Do measures from social choice theory reflect changes in concurrence over the course of deliberation?

As described above, deliberation can potentially change preferences in multiple ways. As social learning, deliberation can change preferences for actions without changing preferences for outcomes, by providing decision-makers with information that allows them to revise their predictions of the efficacy of different actions. Conversely, as discourse, deliberation can change decision-makers’ preferences over outcomes, by building trust and group identity.

RQ2. In deliberation, how much are changes in preferences over actions due to changes in preferences over outcome?

\subsection{Networked Deliberation}
As groups grow in size, deliberation often becomes too cumbersome and focus is shifted to voting (Ackerman \& Fishkin, 2002; Gonz\'alez-Bail\'on, 2010). One possible remedy is to divide large groups into smaller sub-groups. This process allows for the benefits of small-group dynamics, but does not allow information to flow between groups. However, by shuffling groups and repeating the process, it becomes possible for information to diffuse throughout the entire network. The decision-makers and groups form an affiliation network, and I refer to this scheme as networked deliberation. Members might participate in multiple groups simultaneously, or sequentially. This experiment will focus on the sequential case, allowing re-evaluation of preferences between stages. A single stage of the process can also be used to compare the behavior of small groups to large groups.

Def 1. Sequential networked deliberation: N voters are partitioned into groups of size M and allowed to deliberate. The process is repeated T times with new partitions each time.

RQ3. How does the change in concurrence over the course of deliberation depend on group size?

Network structure has been shown to influence social learning (Lazer \& Friedman, 2007; Grim et al., 2013; Barkoczi \& Galesic, 2016), suggesting the importance of understanding the role of network structure in networked deliberation. In networked social learning, the lengths of shortest paths between nodes plays an important role in determining how fast information can travel through a network and the level of functional diversity present in the population (Barkoczi \& Galesic, 2016). The shorter the shortest paths connecting typical nodes, the more efficient, a network is considered.

RQ4a. Does the efficiency of the network topology influence the level of concurrence reached in a networked deliberation?

RQ4b. Does the efficiency of the network topology influence the number of stages necessary to reach concurrence in networked deliberation?

\section{Methods}
\subsection{Quantifying Consensus}
There are various definitions of consensus. Within social choice theory, consensus can be defined by a consensus class, a set of preference profiles meeting one of several consensus criteria (Elkind \& Slinko, 2016). Examples include strong unanimity (of rank orders), unanimity (of winners), majority (existence of majority), Condorcet (existence of Condorcet winner), and transitivity (of social preference). On the most restrictive end, in strong unanimity and unanimity, all decision-makers prefer the same alternative. In contrast, transitivity only requires that the social preferences induced by individual preferences are transitive. Social choice theory also provides measures of distances between two preference rankings which can be extended to measure the distance of a preference profile to a particular consensus class.

One possibility is to measure the distance from strong unanimity. I propose two measures: the mean Kendal tau metric and the mean Spearman correlation. The Kendal tau metric is the number fraction of pairwise contests that have different results between two profiles (Kendall, 1938). By averaging this measure over all pairs of members, an overall concurrence measure can be calculated for the group. This measure does not incorporate information about win/loss margins, only whether an alternative wins or loses. When the margin is important, I propose instead, the mean Spearman correlation. The Spearman correlation is the linear coefficient of correlation between the rank orders of alternatives for two voters (Spearman, 1904). As with Kendall tau, the Spearman correlation can be averaged over all pairs of members to determine a measure of concurrence for the entire group. It must be noted that representing concurrence with a single number will inevitably ignore important information. Neither of these measures distinguishes a small number of voters disagreeing on many comparisons from a large number of voters disagreeing on a few. The Spearman correlation is also unable to distinguish these from situations in which a small number of voters disagree by a very large margin.

While unanimity might be preferred, it may sometimes be more realistic to focus on transitive consensus. This consensus class requires only that the social preferences between pairs of alternatives are transitive. Such preference profiles guarantee that a Condorcet winner will exist, bringing many different voting methods into agreement on which should win.

To measure the distance from transitivity, I propose the number of ranked pair violations. When performing a vote by the Tideman ranked pair system, the relative order of two alternatives in the social preference is determined by the winner of a pairwise vote, except in the special case that contests with a higher margin have already constrained their order due to transitivity. In the latter case, the winner of the pairwise contest may have a lower position in the social preference. If this situation occurs, the social preference is intransitive and there may not be a Condorcet winner. Each such pair represents a contest violating the social ranking, so by counting the fraction of such pairs, we can quantify the strength of intransitivity in the group’s preferences. This measure has the property that a Condorcet winner is guaranteed when the value is 0.

\subsection{Network Topologies}
This experiment requires two network topologies: an efficient topology with short path lengths and an inefficient topology with long path lengths. Both topologies should have the same group size and have relatively little overlap between different groups. The networks are affiliation networks between groups and participants, but can be projected into co-affiliation networks between participants. I will work directly with the co-affiliation projections, in which each group is represented as a fully connected clique of participant nodes. Let N be the number of participants, M be the number of participants in a group, and D be the number of groups a participant belongs to (its degree in the affiliation network).

For the efficient network, I will use a randomized network equivalent to randomly assigning participants to groups in each stage. The nodes are randomly divided into N/M partitions of size M and edges are added between all pairs belonging to the same partition. This process is repeated D times.

To achieve a network with long paths while retaining small overlap between groups, I propose a topology using local subsets of prime residue classes. The prime residue classes guarantee small overlap, while the local subsets ensure long paths. The topology is defined as follows. Let $p_i$ be the $i$th prime number. Let each node be labeled by an integer in $[0, N-1]$. At step $i$, divide the nodes into $p_i$ partitions by their remainder modulo $p_i$. Next, divide each partition into sub-partitions of size $M$, starting with the lowest $M$ node labels, then the next lowest $M$, and so on. Repeat for $i$ in $[0, D-1]$. Note that the sub-partitions of size $M$ connect nodes at most $M*p_i$ apart in label, preventing the creation of any shortcut edges.

Using N = 150, M = 8, D = 3, a typical instance of the efficient topology has diameter 3 and mean shortest path of approximately 2. For the same parameters, the inefficient network has diameter 7 and shortest path 3.17.

\subsection{Experiment}
I propose developing a platform for networked deliberation and conducting an experimental trial in collaboration with a civic organization. The trial will follow the example of the wiki survey case studies conducted in collaboration with the NYC Mayor’s Office and the OECD (Salganik \& Levy, 2015). Members or stakeholders of the organization will be invited to use the online platform to deliberate on proposals for an issue relevant to the organization.

The platform will consist of two components: preference elicitation and deliberation. The preference elicitation component will ask participants to rank order existing proposals, as well as to suggest any additional proposals. Proposals will be ranked graphically using a drag-and-drop interface for ease of use. Participants will have the option of leaving any number of proposals un-ranked, which will be considered tied for the lowest place. Participants will also have the option of writing free-form text arguing for or against each proposal.

The deliberation component will function similarly to a typical online discussion forum. There will be one discussion thread per proposal. Participants will be able to add replies to each proposal and a second level of replies to those replies. Top-level replies will appear chronologically.

Participants will be divided into three groups. One of the most well-studied scales for group size is Dunbar’s number (1992), approximately 150 people. As the goal of this experiment is to evaluate processes for large-scale deliberation, each group will be at least as large as Dunbar’s number. The three groups will be as follows:

Group 1 (control). This group will engage in deliberation between all group members, with sub-group structure. There will be only one stage of deliberation throughout the entire trial.

Group 2 (efficient network). Members of this group will be divided into subgroups of size 8. The trial will be divided into three stages, and members will be reassigned to different subgroups at each stage according to the efficient (randomized) network structure.

Group 3 (inefficient network). Members of this group will be divided into subgroups of size 8. The trial will be divided into three stages, and members will be reassigned to different subgroups at each stage according to the inefficient (local residue class) network structure.

The trial will proceed according to the following stages

Pre-deliberation. Participants are assigned to a group and assigned an id within that group which will determine their sub-groups at each subsequent stage. Participants rank-order existing proposals and suggest any alternative proposals.
Deliberation (3 stages). Participants are shown the deliberation forum, with one thread for each proposal. They are able to see posts and replies only from other members in their current sub-group (as determined by their group’s network structure and the current stage of deliberation). Posts from previous stages remain visible, but cannot be replied to. Each stage of deliberation will last 1 week. After each stage, participants are allowed to change their rank-order preferences and are assigned to a new sub-group for the next round of deliberation.
Post-deliberation and debriefing. Participants have a final chance to revise their rank order preferences. Participants are also asked for any feedback on how the deliberation process has helped them develop their opinions and understand other viewpoints.

\section{Analysis}
Using each measure of the measures of consensus in the Methods section, the concurrence of each group will be tracked over the course of deliberation and differences between various network topologies will be reported. The winning proposals according to various voting methods will also be reported for each group and at each point in the deliberation process.

\section{Discussion}
The proposed experiment has several parameters, including the number of members in each group, the number of members in each sub-group, and the number of stages. The group size has been chosen to be above Dunbar’s number, which represents one scale separating small group behavior from large group behavior. It is possible that there are scales beyond Dunbar’s number which define different regimes for social behavior, but none have been identified. Multiple stages of deliberation are needed to allow participants to belong to multiple sub-groups, but too many stages would be cumbersome for participants. I have chosen 3 stages to balance these two requirements. Similarly, the group size of 8 has been chosen such that the mean shortest path of the efficient network is lower than the number of stages, while that of the inefficient network is higher. Over the course of 3 stages, ideas from any individual could feasibly reach most others in the efficient network, but not for the inefficient network.

There are several possible alternative methods for studying the effect of network structure on deliberation. There are many platforms online where deliberation occurs every day. While a customized platform offers the most potential, existing platforms could also be analyzed. In threaded forums, e.g. Slashdot (Gonz\'alez-Bail\'on, 2010), sub-threads could be analyzed as sub-groups and bots could be used to elicit preferences throughout the deliberation. This method would not be able to determine preferences for all deliberators or at all stages, but could still produce enough information to compare different network structures, to the extent that natural variation exists. In-person events, similar to deliberation days (Ackerman \& Fishkin, 2002), could be used to divide participants into sub-groups according to various network structures for a day of deliberation on a particular topic. An in-person approach has the benefit of keeping participants’ attention over the course of the experiment, with the drawback of requiring physical space and transportation, making large-scale participation more difficult.
